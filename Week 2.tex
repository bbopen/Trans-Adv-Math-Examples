\documentclass[11pt]{article}
\usepackage{amsmath,amssymb,amsthm}
\usepackage{setspace}
\addtolength{\evensidemargin}{-.5in}
\addtolength{\oddsidemargin}{-.5in}
\addtolength{\textwidth}{0.8in}
\addtolength{\textheight}{0.8in}
\addtolength{\topmargin}{-.4in}
\newtheoremstyle{quest}{\topsep}{\topsep}{}{}{\bfseries}{}{ }{\thmname{#1}\thmnote{ #3}.}
\theoremstyle{quest}
\newtheorem*{definition}{Definition}
\newtheorem*{theorem}{Theorem}
\newtheorem*{question}{Question}
\newtheorem*{exercise}{Exercise}
\newtheorem*{answer}{Answer}
\newtheorem*{challengeproblem}{Challenge Problem}
\newcommand{\name}{%%%%%%%%%%%%%%%%%%
%%%%%%%%%%%%%%%%%%%%%%%%%%%%%%
%%%%%%%%%%%%%%%%%%%%%%%%%%%%%%
%% put your name here, so we know who to give credit to %%
Brett Bonner
}%%%%%%%%%%%%%%%%%%%%%%%%%%%%%%
\newcommand{\hw}{%%%%%%%%%%%%%%%%%%%%
%% and which homework assignment is it? %%%%%%%%%
%% put the correct number below              %%%%%%%%%
%%%%%%%%%%%%%%%%%%%%%%%%%%%%%%
1.2 and 1.3
}
%%%%%%%%%%%%%%%%%%%%%%%%%%%%%%
%%%%%%%%%%%%%%%%%%%%%%%%%%%%%%
%%%%%%%%%%%%%%%%%%%%%%%%%%%%%%
\title{\vspace{-50pt}
\Huge \name
\\\vspace{20pt}
\normalsize Math 393 Week 2 Homework Draft 2 -- Section \hw}
\author{}
\date{}
\pagestyle{myheadings}
\markright{\name\hfill Homework \hw\qquad\hfill}

%% If you want to define a new command, you can do it like this:
\newcommand{\Q}{\mathbb{Q}}
\newcommand{\R}{\mathbb{R}}
\newcommand{\Z}{\mathbb{Z}}
\newcommand{\C}{\mathbb{C}}

%% If you want to use a function like ''sin'' or ''cos'', you can do it like this
%% (we probably won't have much use for this)
% \DeclareMathOperator{\sin}{sin}   %% just an example (it's already defined)

\begin{document}
\maketitle
\doublespace
Section 1.2
\begin{exercise}[1]
  Identify the antecedent and the consequent for each of the following 
  conditional sentences. Assume a, b, and f represent some fixed sequence, 
  integer, or function, respectively.
\end{exercise}
\begin{exercise}[1.b]
  ``If the moon is made of cheese, then 8 is an irrational number.''
\end{exercise}

\noindent \textbf{P} antecedent: ``The moon is made of cheese'' \\
\textbf{Q} consequent: ``8 is an irrational number'' \\

\noindent P$\Rightarrow$Q as the antecedent is false, by definition of a conditional 
sentence, the conditional sentence of problem 1.b is true.

\begin{exercise}[1.c]
``b divides 3 only if b divides 9'
\end{exercise}

\noindent \textbf{P} antecedent: ``b divides 3'' \\
\textbf{Q} consequent: ``8 is an irrational number'' \\

\noindent This is a conditional that is equivalent to ``If P, then Q'' \noindent P$\Rightarrow$Q 
\\

\noindent \textit{Additional analysis}: if ``b divides 3'' is true, the set of b consists of elements $\{$-3, -1, 1, 3$\}$ as 3 is prime. If b were to somehow not divide 9 if P is true, then the sentence P$\Rightarrow$Q is false. If ``b divides 3'' is false, then the conditional sentence P$\Rightarrow$Q is true.

\newpage

\begin{exercise}[5]
Which of the following conditional sentences are true?
\end{exercise}
\begin{exercise}[5.b]
  ``If a hexagon has six sides, then the moon is made of cheese.''
\end{exercise}
\noindent \textbf{P} antecedent: ``a hexagon has six sides'' \\
\textbf{Q} consequent: ``the moon is made of cheese'' \\
\noindent An illustration of a hexagon with numerically labeled sides:
\\\vspace{1in}

\noindent Elemental analysis of moon soil samples collected from the 1969 Apollo 
11 mission identify an abundance of silicon and scarcity of calcium relative to 
analysis of cheeses, which supports evidence to the scientific consensus that the moon is not made of 
cheese.\\

\noindent By definition, the conditional sentence P$\Rightarrow$Q if and only if P is 
false or Q is true. The following truth table provides an examination of this

\begin{table}[ht] 
\centering % used for centering table 
\begin{tabular}{c c c } % centered columns (3 columns) 
\hline\hline %inserts double horizontal lines 
P & Q & P$\Rightarrow$Q \\ [0.5ex] % inserts table 
%heading 
\hline % inserts single horizontal line 
T & T & T \\ % inserting body of the table 
F & T & T \\
T & F & F \\
F & F & T \\ [1ex] % [1ex] adds vertical space 
\hline %inserts single line 
\end{tabular} 
\end{table}

\noindent As P is true yet Q is false, P$\Rightarrow$Q is false. The answer to 5.b is 
false.

\newpage

\begin{exercise}[5]
Which of the following conditional sentences are true?
\end{exercise}
\begin{exercise}[5.f]
  ``If Euclid's birthday was April 2, then rectangles have four sides''
\end{exercise}
\noindent \textbf{P} antecedent: ``Euclid's birthday was April 2'' \\
\textbf{Q} consequent: ``rectangles have four sides'' \\

\noindent Euclid may or may not have been born April 2, as little details of Euclid or his life are known by historical scholars\\

\noindent An illustration of a rectangle with numerically labeled sides:
\\\vspace{1in}

By definition, the truth of a conditional P$\Rightarrow$Q is always true when Q 
is true. As a rectangle has for sides, Q is true and therefore the conditional 
statement is true. The answer for 5.f is true.

\newpage

\begin{exercise}[6]
Which of the following are true?
\end{exercise}
\begin{exercise}[6.b]
  ``7+5=12 iff 1+1=2''
\end{exercise}
\noindent \textbf{P}: ``7+5=12'' is true $\forall$ $\mathbb{R}$ \\
\textbf{Q}: ``1+1=2'' is true $\forall$ $\mathbb{R}$ \\
P$\Leftrightarrow$Q

Examining the truth table for a biconditional sentence will help assess the 
answer

\begin{table}[ht] 
\centering % used for centering table 
\begin{tabular}{c c c} % centered columns (3 columns) 
\hline\hline %inserts double horizontal lines 
P & Q & P$\Leftrightarrow$Q \\ [0.5ex] % inserts table 
%heading 
\hline % inserts single horizontal line 
T & T & T \\ % inserting body of the table 
F & T & F \\
T & F & F \\
T & F & F \\
F & F & T \\ [1ex] % [1ex] adds vertical space 
\hline %inserts single line 
\end{tabular} 
\end{table}

As P is true and Q is true, P$\Leftrightarrow$Q is true. The answer for 6.b is 
true.

\begin{exercise}[6.h]
  `` $\sqrt{10}$ + $\sqrt{13}$ $<$ $\sqrt{11}$ + $\sqrt{12}$ iff $\sqrt{13}$ - $\sqrt{12}$ $<$ $\sqrt{11}$ - $\sqrt{10}$ ''
\end{exercise}
\noindent \textbf{P}: `` $\sqrt{10}$ + $\sqrt{13}$ $<$ $\sqrt{11}$ + $\sqrt{12}$ '' is true as $\sqrt{10}$ + $\sqrt{13}$ $\approx$ 6.767, $\sqrt{11}$ + $\sqrt{12}$ $\approx$ 6.780 \\
\textbf{Q}: `` $\sqrt{13}$ - $\sqrt{12}$ $<$ $\sqrt{11}$ - $\sqrt{10}$ '' is true as $\sqrt{13}$ - $\sqrt{12}$ $\approx$ 0.141, $\sqrt{11}$ - $\sqrt{10}$ $\approx$ 0.154 \\
P$\Leftrightarrow$Q

As ``iff'' can be inferred as a biconditional connective, P$\Leftrightarrow$Q, 
the sentence is true as P and Q have the same truth values. The answer for 6.h 
is true.

\begin{exercise}[6.i]
  `` $x_2$$\ge$0 iff $x$$\ge$0 (assume $x$ is a fixed real number) '' 
\end{exercise}
\noindent \textbf{P}: `` $x_2$$\ge$0 '' is true as the product of any real number and itself is either positive or 0 \\
\textbf{Q}: `` $x$$\ge$0 '' cannot be assessed as $x$ being a fixed real number could be negative. \\
As P is always true but Q may be false, P$\Leftrightarrow$Q is false and the 
answer to 6.i is false.

\newpage

\begin{exercise}[8]
Prove Theorem 1.2.2 by constructing truth tables for each equivalence
\end{exercise}
\begin{exercise}[8.c]
  ``$\neg$ (P$\Rightarrow$Q) is equivalent to P$\wedge$$\neg$Q''
\end{exercise}

Examining the truth table will prove the two sentences are equivalent, as their 
resulting truth values are identical given the same arguments of P, Q

\begin{table}[ht] 
\centering % used for centering table 
\begin{tabular}{c c c c c c} % centered columns (3 columns) 
\hline\hline %inserts double horizontal lines 
P & Q & $\neg$Q & P$\Rightarrow$Q & $\neg$(P$\Rightarrow$Q) & P$\wedge$$\neg$Q \\ [0.5ex] % inserts table 
%heading 
\hline % inserts single horizontal line 
T & T & F & T & F & F \\ % inserting body of the table 
T & F & T & F & T & T \\
F & T & F & T & F & F \\
F & F & T & T & F & F \\ [1ex] % [1ex] adds vertical space 
\hline %inserts single line 
\end{tabular} 
\end{table}

$\neg$ (P$\Leftarrow$Q) negates the result of ``P$\Leftrightarrow$Q if and only if P is false or Q is true.' 
Therefore, $\neg$(P$\Rightarrow$Q) is true if and only if P is true and Q is false.\\
Accordingly, P$\wedge$$\neg$Q is true if and only if P is true and Q is false.

\begin{exercise}[8.d]
  ``$\neg$(P$\wedge$Q) is equivalent to P$\Rightarrow$$\neg$Q and to Q$\Rightarrow$$\neg$P''
\end{exercise}

Examining the truth table will prove the three sentences are equivalent, as their 
resulting truth values are identical given the same arguments of P, Q

\begin{table}[ht] 
\centering % used for centering table 
\begin{tabular}{c c c c c c c c} % centered columns (3 columns) 
\hline\hline %inserts double horizontal lines 
P & $\neg$P & Q & $\neg$Q & P$\wedge$Q & $\neg$(P$\wedge$Q) & P$\Rightarrow$$\neg$Q & Q$\Rightarrow$$\neg$P\\ [0.5ex] % inserts table 
%heading 
\hline % inserts single horizontal line 
T & F & T & F & T & F & F & F \\ % inserting body of the table 
T & F & F & T & F & T & T & T \\
F & T & T & F & F & T & T & T \\
F & T & F & T & F & T & T & T \\ [1ex] % [1ex] adds vertical space 
\hline %inserts single line 
\end{tabular} 
\end{table}

$\neg$(P$\wedge$Q) negates the result of ``P$\wedge$Q if and only if P and Q are true.' 
Therefore $\neg$(P$\wedge$Q) is true if P or Q are false.\\
Accordingly, P$\Rightarrow$$\neg$Q is true if P or Q are false, which is equivalent to Q$\Rightarrow$$\neg$P

\newpage

\begin{exercise}[14]
  `` Give, if possible, an example of a false conditional sentence for which ''
\end{exercise}
\begin{exercise}[14.a]
  `` The converse is true ''
\end{exercise}

\noindent \textbf{P$\Rightarrow$Q}: `` If the sum of all interior angles of a shape = $180\,^{\circ}$, then the shape is a triangle \\
\noindent The converse \textbf{Q$\Rightarrow$P}: `` If a shape is a triangle, then the sum of all interior angles of a shape = $180\,^{\circ}$.
\begin{exercise}[14.b]
  `` The converse is false ''
\end{exercise}
\noindent It is not possible for the converse of a false statement to be false. A conditional statement is false if and only if the antecedent is true and the consequent is false. In the converse, when the false consequent becomes the antecedent in a new statement, the new statement is true: because a false antecedent is always a true conditional statement.

\begin{exercise}[14.c]
  `` The contrapositive is false ''
\end{exercise}
\noindent \textbf{P$\Rightarrow$Q}: `` If $x$ is a natural number of $\mathbb{N}_1$, then $x^2=0$' is false because 0 is not a natural in $\mathbb{N}_1$ \\
\noindent The contrapositive \textbf{Q$\Rightarrow$P}: `` If $x^2$$\neq$$0$, then x is not a natural number of $\mathbb{N}_1$ '' can be false as well when $x$$\in$$\mathbb{N}_1$

\begin{exercise}[14.d]
  `` The contrapositive is true ''
\end{exercise}
\noindent If a conditional sentence is false, it is not possible for a contrapositive to be true. Examining the following truth table provides proof: \\ 
\begin{table}[ht] 
\centering % used for centering table 
\begin{tabular}{c c c c c c} % centered columns (3 columns) 
\hline\hline %inserts double horizontal lines 
P & $\neg$P & Q & $\neg$Q & P$\Rightarrow$Q & ($\neg$Q)$\Rightarrow$($\neg$P) \\ [0.5ex] % inserts table 
%heading 
\hline % inserts single horizontal line 
T & F & T & F & T & T \\ % inserting body of the table 
\textbf{T} & \textbf{F} & \textbf{F} & \textbf{T} & \textbf{F} & \textbf{F} \\
F & T & T & F & T & T \\
F & T & F & T & T & T \\ [1ex] % [1ex] adds vertical space 
\hline %inserts single line 
\end{tabular} 
\end{table}

\newpage

Section 1.3
\begin{exercise}[1]
  `` Translate the following English sentences into symbolic sentences with quantifiers.  ''
\end{exercise}
\begin{exercise}[1.c]
  `` Some isosceles triangles is a right triangle (all triangles) ''
\end{exercise}
$(\exists x)$$(x$ is isosceles $\wedge$ $x$ is right)\\
``There exists a triangle such that the triangle is isosceles and is right.''

\begin{exercise}[1.f]
  `` Some people are honest and some people are not honest (all people) ''
\end{exercise}
$(\exists x)$$(x$ is honest) $\wedge$ $(\exists x)$$(x$ is $\neg$honest) \\
``There exists a person such that person is honest and there exists a person such that a person is not honest.''

\newpage

\begin{exercise}[7]
  `` Complete this proof of theorem 1.31 ''
\end{exercise}
\begin{exercise}[7.a]
  `` Let $\mathbb{U}$ be any universe. The sentence ($\exists$$x$)$A(x)$ is true in $\mathbb{U}$\\
  iff...\\
  iff ($\forall$$x$)$\not$A$(x)$ is true in $\mathbb{U}$ ''
\end{exercise}
Let $\mathbb{U}$ be any universe. If A$(x)$ is an open sentence with variable 
$x$, then $\neg$($\exists$$x$)A($x$) is equivalent to ($\forall$$x$)$\neg$A($x$) 
in $\mathbb{U}$ iff ($\exists$$x$)A($x$) is false in $\mathbb{U}$, iff the truth 
set of $\neg$A($x$) is nonempty and the entire universe of $\mathbb{U}$ and iff 
($\forall$$x$)$\neg$A($x$) is true in $\mathbb{U}$

\newpage

\begin{exercise}[10]
  `` Which of the following are true in the universe of real numbers? ''
\end{exercise}
\begin{exercise}[10.f]
  `` ($\exists$$x$) ($\forall$$y$) ($x$$\le$$y$)''
\end{exercise}
\noindent ``There exists a real number $x$ such that for all real numbers $y$, 
$x$$\le$$y$''\\
\textbf{False}. As set of all real numbers is uncountably infinite and the real line being a linear continuum with no maximum or minimum element, there can not be a $x$ outside of all real numbers $y$ that is less than $y$.

\begin{exercise}[10.g]
  `` ($\forall$$y$)($\exists$$x$)($x$$\le$$y$)''
\end{exercise}
\noindent ``For all real numbers $y$, there exists a real number $x$ such that $x$ is less than or equal to $y$.''\\
\textbf{True}. $x$ being within the set of all $y$ would mean that given any real number in $y$, there can be an $x$ such that it is equal to or less than $y$.

\newpage

\begin{exercise}[12.a]
  `` Write the symbolic form for the definition of `f is continuous at a'. ''
\end{exercise}

\noindent Using the Weierstrass definition of a continuous function. ``For any real number 
$\epsilon$$\textgreater$0, there exists some number $\delta$$\textgreater$0 such that for all real numbers 
$x$ with the domain of $f$, if $|$$x-a$$|$ $\textless$ $\delta$ then $|$$f(x)-f(a)$$|$ $\textless$ $\epsilon$ ''\\

($\forall$ $\epsilon$ $\textgreater$ 0) ($\exists$ $\delta$ $\textgreater$ 0) ($\forall$ $x$)[ $(|$$x-a$$|$ $\textless$ $\delta$) $\Rightarrow$ ( $|$$f(x)-f(a)$$|$ $\textless$ $\epsilon$)]

\begin{exercise}[12.d]
  `` Write a useful denial for 12.a ''
\end{exercise}

\noindent ``There exists a real number $\epsilon$ $\textgreater$ 0, for any real number $\delta$ $\textgreater$ 0 there 
exists an $x$ with the domain of $f$, $|$$x-a$$|$ $\textless$ $\delta$ and $|$$f(x)-f(a)$$|$ $\ge$ $\epsilon$ ''\\

($\exists$ $\epsilon$ $\textgreater$ 0) ($\forall$ $\delta$ $\textgreater$ 0) ($\exists$ $x$) [$(|$$x-a$$|$ $\textless$ $\delta$) $\wedge$ ( $|$$f(x)-f(a)$$|$ $\ge$ $\epsilon$)]


\newpage

\begin{exercise}[PPT Problem]
  `` Suppose that A($x$) is an open sentence with variable $x$. Write a statement (similar to the one in (b) of the theorem) to express the fact that there exist exactly two distinct elements that have property 
  A.''\\
  \textbf{Theorem 1.3.2 B}:\\ ($\exists$$!$$x$)(A($x$)) is equivalent to 
  ($\exists$$x$)(A($x$))$\wedge$($\forall$$y$)($\forall$$z$)[A($y$)$\wedge$A($z$)$\Rightarrow$$y=z$]\\
\end{exercise}

\noindent There exist two numbers $x$ and $y$: ($\exists$$x$) ($\exists$$y$)\\
with property A: $A(x)$$\wedge$$A(y)$ and \\
$x$ does not equal $y$: $x$$\neq$$y$ and \\
and if there is a number $z$ with property A, then it must be $x$ and $y$:\\ 
$A(z)$$\Rightarrow$[($z=x$)$\wedge$($z=y$)]\\
($\exists$$x$) ($\exists$$y$) [($A(x)$$\wedge$$A(y)$) $\wedge$ ($x$$\neq$$y$) $\wedge$ [($\forall$$z$)($A(z)$$\Rightarrow$[ ($z=x$)$\wedge$ ($z=y$)]]]

\end{document}
