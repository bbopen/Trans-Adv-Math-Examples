\documentclass[a4paper,11pt]{article}
\usepackage{amsmath,amsthm,amssymb}
\usepackage{mathtools}
\usepackage{mathrsfs}
\usepackage{setspace}
\usepackage{enumerate}
\usepackage{caption}
\DeclarePairedDelimiter\abs{\lvert}{\rvert}
\begin{document}
\newtheorem*{theorem1}{Theorem}
\newtheorem*{theorem2}{Theorem}
\newtheorem*{theorem3}{Theorem}
\newtheorem*{theorem4}{Theorem}
\newtheorem*{theorem5}{Theorem}
\newtheorem*{theorem6}{Theorem}
\newtheorem*{theorem7}{Theorem}
\newtheorem*{theorem8}{Theorem}
\newtheorem*{theorem9}{True/False?}
\title{MATH 393 Week 7 Assignment 2\textsuperscript{nd} Draft}
\author{Brett Bonner}
\date{March 10, 2014}
\maketitle
\linespread{1.5}
\doublespacing
\newcounter{ProblemCounter}
\newcounter{SubsectionCounter}[ProblemCounter]
\addtocounter{ProblemCounter}{6} % set them to some other numbers than 0
\addtocounter{SubsectionCounter}{3} % same
%

\section*{\S 2.4 Exercise \arabic{ProblemCounter}: Use PMI to prove the following for all natural numbers \(n\).}
\textbf{\arabic{ProblemCounter}.\alph{SubsectionCounter}}
\(\sum\limits_{i=1}^n 2^i = 2^{n+1} - 2\)
\begin{theorem1}
  \(\sum\limits_{i=1}^n 2^i = 2^{n+1} - 2 \) for all natural numbers \(n\).
\begin{proof}
  Let \(S = \{n \in \mathbb{N}: \sum\limits_{i=1}^n 2^i = 2^{n+1} - 2 \}\)\\
  \begin{enumerate}[(i)]
    \item  To show the statement is true for \(n=1\).
    \begin{align*}
    &\sum\limits_{i=1}^1 2^i = 2^1 = 2\\
    &2^{n+1}-2 = 2^{1+1} -2 = 2^{2} -2 = 4 - 2 = 2
    \end{align*}
    Therefore \(1 \in S\).
    \newpage
    \item Assume that \(\sum\limits_{i=1}^n 2^i = 2^{n+1} - 2 \) for all natural numbers 
    \(n\).\\
    Using the hypothesis of induction, to show that if \(n \in S\), then \(n+1 \in S\) for all natural numbers \(n\):
    \begin{align*}
      \sum\limits_{i=1}^{n+1} 2^i &= 2^{{(n+1)}+1} - 2\\
      \text{Rewriting the summation, }2^{n+1} + \sum\limits_{i=1}^{n} 2^i &=2^{n+2} - 2\\
      \text{Using the assumption }\sum\limits_{i=1}^n 2^i = 2^{n+1} - 2, 2^{n+1} + {(2^{n+1} -2)} &= 2^{n+2} - 2\\
      2^{n+2} - 2 &= 2^{n+2} - 2
    \end{align*}
    Therefore 
    \item By (i), (ii), and PMI, we showed that \(1 \in S\) and if \(n \in S\), then \(n+1 \in S\) 
    and therefore \(\sum\limits_{i=1}^n 2^i = 2^{n+1} - 2 \) for all natural numbers \(n\).
  \end{enumerate}
\end{proof}  
\end{theorem1}
\newpage

\setcounter{SubsectionCounter}{5}
\section*{\S 2.4 Exercise \arabic{ProblemCounter}: Use PMI to prove the following for all natural numbers \(n\).}
\textbf{\arabic{ProblemCounter}.\alph{SubsectionCounter}}
\(1^3 + 2^3 + \cdots + n^3 = \left[\frac{n{(n+1)}}{2}\right]^2 \)
\begin{theorem1}
  \(1^3 + 2^3 + \cdots + n^3 = \left[\frac{n{(n+1)}}{2}\right]^2 \) for all natural numbers \(n\).
\begin{proof}
  Let \(S = \{n \in \mathbb{N}: 1^3 + 2^3 + \cdots + n^3 = \left[\frac{n{(n+1)}}{2}\right]^2 \}\)\\
  \begin{enumerate}[(i)]
    \item  To show \(1^3 + 2^3 + \cdots + n^3 = \left[\frac{n{(n+1)}}{2}\right]^2\) is true for 
    \(n=1\):
    \begin{align*}
    1^3 &= \left[\frac{1{(1+1)}}{2}\right]^2\\
    1 &= \left[\frac{{(2)}}{2}\right]^2\\
    1 &= 1
    \end{align*}
    Therefore \(1 \in S\).
    \newpage
    \item To show if \(n \in S\), then \(n+1 \in S\) for all natural numbers \(n\),\\
     assume \(1^3 + 2^3 + \cdots + n^3 = \left[\frac{n{(n+1)}}{2}\right]^2 \).\\
     We want to show that \(1^3 + 2^3 + \cdots + {(n+1)}^3 = \left[\frac{{(n+1)}{(n+1+1)}}{2}\right]^2\) 
    is true:
    \begin{align*}
    1^3 + 2^3 + \cdots + n^3 + {(n+1)^3} &= \left[\frac{{(n+1)}{(n+1+1)}}{2}\right]^2\\
    \text{We substitute with our assumption }\left[\frac{n{(n+1)}}{2}\right]^2 + {(n+1)}^3 &= \left[\frac{{(n+1)}{(n+2)}}{2}\right]^2\\
    \frac{n^2{(n+1)^2}}{2^2} + {(n+1)}^3 &= \cdots\\
    \frac{n^2{(n+1)^2}+4{(n+1)}^3}{4} &= \cdots\\
    \frac{{(n+1)}^2{(4n+4)}{(n^2)}}{4} &= \cdots\\
    \frac{{(n+2)}{(n+2)}{(n+1)}^2}{4} &= \cdots\\
    \frac{{(n+2)}^2{(n+1)}^2}{4} &= \cdots\\
    \left[\frac{{(n+1)}{(n+2)}}{2}\right]^2 &= \left[\frac{{(n+1)}{(n+2)}}{2}\right]^2
    \end{align*}
    \item By {(i)} and {(ii)} and PMI, we showed that \(1 \in S\) and if \(n \in S\), then \(n+1 \in S\) 
    and therefore \(1^3 + 2^3 + \cdots + n^3 = \left[\frac{n{(n+1)}}{2}\right]^2 \) for all natural numbers \(n\).
\end{enumerate}
\end{proof} 
\end{theorem1}

\newpage

\setcounter{SubsectionCounter}{9}
\section*{\S 2.4 Exercise \arabic{ProblemCounter}: Use PMI to prove the following for all natural numbers \(n\).}
\textbf{\arabic{ProblemCounter}.\alph{SubsectionCounter}}
\(\sum\limits_{i=1}^n\frac{1}{{(2i-1){(2i+1)}}}=\frac{n}{2n+1}\)
\begin{theorem1}
  \(\sum\limits_{i=1}^n\frac{1}{{(2i-1){(2i+1)}}}=\frac{n}{2n+1}\) for all natural numbers n.
\begin{proof}
  Let \(S = \{n \in \mathbb{N}: \sum\limits_{i=1}^n\frac{1}{{(2i-1){(2i+1)}}}=\frac{n}{2n+1}\}\)
  \begin{enumerate}[(i)]
    \item To show that \(1 \in S\):
    \begin{align*}
      \sum\limits_{i=1}^1\frac{1}{{(2(1)-1){(2(1)+1)}}} &= \frac{1}{2(1)+1}\\
      \frac{1}{{(1)}{(3)}} &= \frac{1}{2+1}\\
      \frac{1}{3} &= \frac{1}{3}
    \end{align*}
    \newpage
    \item To show if \(n \in S\) then \(n+1 \in S\), assume \(\sum\limits_{i=1}^n\frac{1}{{(2i-1){(2i+1)}}}=\frac{n}{2n+1}\):
    \begin{align*}
      \sum\limits_{i=1}^{n+1}\frac{1}{{(2i-1){(2i+1)}}} &= 
      \frac{n+1}{2{(n+1)}+1}\\
      \frac{1}{{(2{(n+1)}-1)}{(2{(n+1)}+1)}} + \sum\limits_{i=1}^{n}\frac{1}{{(2i-1){(2i+1)}}} 
      &= \frac{n+1}{2n+3}\\
      \frac{1}{{(2n+1)}{(2n+3)}} + \sum\limits_{i=1}^{n}\frac{1}{{(2i-1){(2i+1)}}} &= 
      \frac{n+1}{2n+3}\\
      \frac{1}{{(2n+1)}{(2n+3)}} + \frac{n}{2n+1} &= \frac{n+1}{2n+3}\\
      \frac{1+n{(2n+3)}}{{(2n+1)}{(2n+3)}} &= \frac{n+1}{2n+3}\\
      \frac{{(2n+1)}{(n+1)}}{{(2n+1)}{(2n+3)}} &= \frac{n+1}{2n+3}\\
      \frac{n+1}{2n+3} &= \frac{n+1}{2n+3}
    \end{align*}
    Therefore \(n+1 \in S\).
    \item By (i) and (ii) and PMI, we showed that \(1 \in S\) and if \(n \in S\), then \(n+1 \in S\) and therefore  \(\sum\limits_{i=1}^n\frac{1}{{(2i-1){(2i+1)}}}=\frac{n}{2n+1}\) for all natural numbers n..
  \end{enumerate}
\end{proof}  
\end{theorem1}

\newpage

\setcounter{ProblemCounter}{8}
\setcounter{SubsectionCounter}{2}
\section*{\S 2.4 Exercise \arabic{ProblemCounter}: Use the Generalized PMI to prove the following.}
\textbf{\arabic{ProblemCounter}.\alph{SubsectionCounter}}
\(2^n > n^2\) for all \(n > 4\).
\begin{theorem1}
  \(2^n > n^2\) for all \(n > 4\).
\begin{proof}
  Let \(S=\{n > 4: 2^n > n^2\}\)
  \begin{enumerate}[(i)]
    \item Show that \(5 \in S\) where \(S = \{n >4 | 2^n > n^2\}\)\\
    \(2^5 > 5^2\)\\
    \(32 > 5^2\)\\
    \(32 > 25\), \text{ which is true and therefore } \(5 \in S\)
    \newpage
    \item Assume \(n \in S\) which means \(2^n > n^2\)\\
    Show that \(n+1 \in S\) which means \(2^{n+1} > {(n+1)}^2\)
    \begin{align*}
      2^{n+1} &= 2{(2^n)}\\
      &> 2n^2 \text{ by the induction hypothesis}\\
      &= n^2 + n^2\\
      &> n^2 + 4n \text{ because } n > 4 \Leftrightarrow n^2 > 4n\\
      &= n^2 + 2n + 2n\\
      &> n^2 +2n + 8 \text{ because } n > 4 \Leftrightarrow 2n > 8\\
      &> n^2 +2n +1\\ 
      &> n^2 + 2n + 1 \text{ because } 8 > 1\\
      &= {(n+1)}^2
    \end{align*}
    \item By {(i)} and {(ii)} the Generalized PMI we proved  \(2^n > n^2\) for all natural numbers \(n > 4\)
  \end{enumerate}
\end{proof}  
\end{theorem1}

\newpage

\setcounter{ProblemCounter}{8}
\setcounter{SubsectionCounter}{5}
\section*{\S 2.4 Exercise \arabic{ProblemCounter}: Use the Generalized PMI to prove the following.}
\textbf{\arabic{ProblemCounter}.\alph{SubsectionCounter}}
\(\prod\limits_{i=2}^n \frac{i^2-1}{i^2} = \frac{n+1}{2n}\) for all \(n \geq 2\).
\begin{theorem1}
\(\prod\limits_{i=2}^n \frac{i^2-1}{i^2} = \frac{n+1}{2n}\) for all \(n \geq 2\).
\begin{proof}
    \begin{enumerate}[(i)]
    \item To show \(2 \in n\)
    \begin{align*}
      \prod\limits_{i=2}^2 \frac{2^2-1}{2^2} &= \frac{2+1}{2{(2)}}\\
      \frac{3}{4} &= \frac{3}{4}
    \end{align*}
    Therefore \(2 \in n\).
    \newpage
    \item To show if \(n \in S\) then \(n+1 \in S\), assume \(\prod\limits_{i=2}^n \frac{i^2-1}{i^2} = 
    \frac{n+1}{2n}\). Then\\
    \begin{align*}
      \prod\limits_{i=2}^{n+1} \frac{i^2-1}{i^2} &= 
    \frac{{(n+1)}+1}{2{(n+1)}}\\
    \frac{{(n+1)}^2-1}{{(n+1)}^2} \cdot \prod\limits_{i=2}^n \frac{i^2-1}{i^2} 
    &= \frac{n+2}{2n+2}\\
    \frac{{(n+1)}^2-1}{{(n+1)}^2} \cdot \frac{n+1}{2n} 
    &= \cdots \text{ by the induction hypothesis}\\
    \frac{({(n+1)}^2-1){(n+1)}}{2n(n+1)^2} &= \cdots\\
    \frac{({(n+1)}^2-1){(n+1)}^{-1}}{2n} &= \cdots\\
    \frac{{(n+1)}^2-1^2}{2n{(n+1)}} &= \cdots\\
    \frac{{(1+n-1)}{(n+1+1)}}{2n{(n+1)}} &= \cdots\\
    \frac{n{(n+2)}}{2n{(n+1)}} &= \cdots\\
    \frac{n+2}{2n+2} &= \frac{n+2}{2n+2}
    \end{align*}
    \item By {(i)} and {(ii)} and Generalized PMI, we showed that \(2 \in n\) 
    and if \(n \in S\), then \(n+1 \in S\) and therefore \(\prod\limits_{i=2}^n \frac{i^2-1}{i^2} = \frac{n+1}{2n}\) for all \(n \geq 
    2\).
  \end{enumerate}
\end{proof}  
\end{theorem1}

\newpage

\setcounter{ProblemCounter}{11}
\section*{\S 2.4 Exercise \arabic{ProblemCounter}: Towers of Hanoi: Use the PMI to prove that with \(n\) disks, the puzzle can be solved in \(2^n-1\) moves.}

\begin{theorem1}
With \(n\) disks, the Towers of Hanoi puzzle can be solved in \(2^n-1\) moves.
\begin{proof}
  Let \(S = \{n | \text{For } n \text{ disks, it takes } 2^n-1 \text{ moves to complete the puzzle}\}\)
    \begin{enumerate}[(i)]
  \item To show \(1 \in S\), let \(n=1\).\\
  \(2^{(1)}-1 = 2 - 1 = 1\). Therefore \(1 \in S\)
  \begin{figure}
  [htbp]
\centering
\includegraphics[width=1.0\textwidth]{1.png}
\caption*{}
\label{fig:1}
\end{figure}
\begin{figure}[htbp]
\centering
\includegraphics[width=1.0\textwidth]{2.png}
\caption*{For 1 disk, \(2^1-1 = 1\)}
\label{fig:2}
\end{figure}
\newpage
 \item Assume \(n \in S\) and that for a puzzle with \(n\) disks, it will take \(2^n-1\) 
 moves to complete the puzzle.\\
 To show \(n+1 \in S\), a puzzle with \(n+1\) disks can be solved in \(2^{n+1}-1\)
\begin{figure}[h!]
\centering
\includegraphics[width=1.0\textwidth]{6.png}
\label{fig:6}
\end{figure}
\begin{figure}[h!]
\centering
\includegraphics[width=1.0\textwidth]{3.png}
\caption*{It takes \(2^n-1\) disks to move the top \(n\) disks to the center}
\label{fig:3}
\end{figure}
\begin{figure}[h!]
\centering
\includegraphics[width=1.0\textwidth]{4.png}
\caption*{It takes an additional move to place the \({(n+1)}^{\text{nd}}\) disk to the rightmost rod}
\label{fig:4}
\end{figure}
\begin{figure}[h!]
\centering
\includegraphics[width=1.0\textwidth]{5.png}
\caption*{It will take another \(2^n-1\) moves to place \(n\) disks from the center rod to the rightmost rod}
\label{fig:5}
\end{figure}
\\The total number of moves to complete the puzzle:\\
\(2^{n}-1+1+2^{n}-1 = 2^{n+1} - 1\) and therefore if \(n \in S\), then \(n+1 \in 
S\).
\item By {(i)} and {(ii)} and PMI, we proved that \(1 \in S\) and that if \(n \in 
S\), then \(n+1 \in S\) and therefore we proved with \(n\) disks, the Towers of Hanoi puzzle can be solved in \(2^n-1\) 
moves..
\end{enumerate}
\end{proof}  
\end{theorem1}

\newpage

\setcounter{ProblemCounter}{4}
\section*{\S 2.5 Exercise \arabic{ProblemCounter}: Fibonacci problem:}
\setcounter{SubsectionCounter}{1}
\textbf{\arabic{ProblemCounter}.\alph{SubsectionCounter}}
Show that at \(n\) months, there are \(f_n\) pairs of rabbits.
\begin{theorem1}
 At \(n\) months, there are \(f_n\) pairs of rabbits.
\begin{proof}
Assume Fibonacci's rabbits are in fact the killer Rabbits of Caerbannog from 
\textit{Monty Python and the Holy Grail} and that they never die naturally (as only the Holy Hand Grenade of Antioch may kill them). Despite being 
apex predators of ultimate death, it takes one month for these rabbits to get their 
mojo working and another month to produce offspring. The offspring will always be one pair, a male and a female. Help us all!\\
As we start with one pair, the population in first two months will thankfully be limited 
to one pair of the beasts. After month three, however, an additional pair will emerge from the Cave of Caerbannog. As these deadly creatures 
are only spawned from parent rabbits alive in the prior two months, \({n-2}\), and 
the population from the prior month \(n-1\) are unfortunately still alive. So
the total population of killer Rabbits of Caerbannog in any month \(n\) is the prior population of rabbits plus the new rabbits of the month, or \(f_n = 
f_{n-1}+f_{n_2}\). Bring out your dead!
\end{proof}  
\end{theorem1}

\newpage

\setcounter{ProblemCounter}{4}
\section*{\S 2.5 Exercise \arabic{ProblemCounter}: Fibonacci problem:}
\setcounter{SubsectionCounter}{2}
\textbf{\arabic{ProblemCounter}.\alph{SubsectionCounter}}
Calculate the first ten Fibonacci numbers \(f_1,f_2,f_3,\ldots,f_{10}\).
\begin{theorem1}
 The first ten Fibonacci numbers are found by calculating according to the definition of the Fibonacci 
 sequence:\\
\(f_n = f_{n-1} + f_{n-2}\), with the seed values of \(f_1 = 1\) and \(f_2 = 
1\)\\
\begin{alignat*}{7}
& & & & & & f_1 &= 1\\
& & & & & & f_2 &= 2\\
f_3 &= f_{3-1} &+ f_{3-2} &= f_{2} &+ f_{1} &= 1 &+ 1 &= 2\\
f_4 &= f_{4-1} &+ f_{4-2} &= f_{3} &+ f_{2} &= 2 &+ 1 &= 3\\
f_5 &= f_{5-1} &+ f_{5-2} &= f_{4} &+ f_{3} &= 3 &+ 2 &= 5\\
f_6 &= f_{6-1} &+ f_{6-2} &= f_{5} &+ f_{4} &= 5 &+ 3 &= 8\\
f_7 &= f_{7-1} &+ f_{7-2} &= f_{6} &+ f_{5} &= 8 &+ 5 &= 13\\
f_8 &= f_{8-1} &+ f_{8-2} &= f_{7} &+ f_{6} &= 13 &+ 8 &= 21\\
f_9 &= f_{9-1} &+ f_{9-2} &= f_{8} &+ f_{7} &= 21 &+ 13 &= 34\\
f_{10} &= f_{10-1} &+ f_{10-2} &= f_{9} &+ f_{8} &= 34 &+ 21 &= 55\\
\end{alignat*}
Therefore the set of the first ten Fibonacci numbers is \{1,1,2,3,5,8,13,21,34,55\}
\end{theorem1}

\newpage

\setcounter{ProblemCounter}{4}
\section*{\S 2.5 Exercise \arabic{ProblemCounter}: Fibonacci problem:}
\setcounter{SubsectionCounter}{3}
\textbf{\arabic{ProblemCounter}.\alph{SubsectionCounter}}
Find a formula for \(f_{n+3}-f_{n+1}\).
\begin{theorem1}
 A formula for \(f_{n+3}-f_{n+1}\) is found by:
 \begin{proof}
The Fibonacci sequence is defined \(f_n = f_{n-1} + f_{n-2}\), with the seed values of \(f_1 = 1\) and \(f_2 = 
1\).\\
\begin{align*}
  f_{n+3}-f_{n+1} &= {(f_{{(n-1)+3}}+f_{{(n-2)}+3})} - f_{n+1}\\
  \ldots &= f_{n+2}+f_{n+1} - f_{n+1}\\
  f_{n+3}-f_{n+1} &= f_{n+2}
\end{align*}
Therefore a formula for \(f_{n+3}-f_{n+1}\) is \(f_{n+2}\).
\end{proof}  
\end{theorem1}

\newpage

\setcounter{ProblemCounter}{6}
\section*{\S 2.5 Exercise \arabic{ProblemCounter}: Use the PCI to prove the following properties of Fibonacci numbers:}
\setcounter{SubsectionCounter}{2}
\textbf{\arabic{ProblemCounter}.\alph{SubsectionCounter}}
\(f_{n+6}=4f_{n+3}+f_n\) for all natural numbers \(n\).
\begin{theorem1}
\(f_{n+6}=4f_{n+3}+f_n\) for all natural numbers \(n\).
 \begin{proof}
  The first terms of the Fibonacci series are \{1,1,2,3,5,8,13,21,34,\ldots\}\\
  Let \(S = \{n \in \mathbb{N} | f_{n+6}=4f_{n+3}+f_{n}\}\)
  \begin{enumerate}[(i)]
    \item To show \(1 \in S\),
    \begin{align*}
      f_{1+6} &= 4f_{1+3}+f_{1}\\
      f_{7} &= 4f_{4}+f_{1}\\
      \text{Matching } f_n \text{ to corresponding}&\\
      \text{elements of the Fibonacci sequence} \ldots&\\
      13 &= 4(3) + 1\\
      13&=13
    \end{align*}
    To show \(2 \in S\),
    \begin{align*}
      f_{2+6} &= 4f_{2+3}+f_{2}\\
      f_{8} &= 4f_{5}+f_{2}\\
      \text{Matching } f_n \text{ to corresponding}&\\
      \text{elements of the Fibonacci sequence} \ldots&\\
      21 &= 4(5) + 1\\
      21&=21
    \end{align*}
    Therefore \(1 \in S\) and \(2 \in S\)
    \newpage
    \item To prove if \(n \in S\), assume that \(1, 2, 3, \ldots n-1 \in S\).\\
    To compose \(f_{n+6}\), we use the definition of the Fibonacci sequence\\
    \(f_{n+6} &= f_{n+5}+f_{n+4}\)\\
    To compose \(f_{n+5}\), \(f_{{(n-1)}+6} &= 4f_{{(n-1)+3}}+f_{(n-1)} = 4f_{n+2}+f_{n-1}\)\\
    To compose \(f_{n+4}\), \(f_{{(n-2)}+6} &= 4f_{{(n-2)+3}}+f_{(n-2)} = 4f_{n+1}+f_{n-2}\)\\
    Therefore \(f_{n+6}=4f_{n+2}+f_{n-1}+4f_{n+1}+f_{n-2}\)\\
    \(f_{n+6}=4{(f_{n+2}+f_{n+1})}+{(f_{n-2}+f_{n-1})}\)\\
    \(f_{n+6}=4{(f_{n+3})}+{f_n}\)\\
    \item This proves \(n \in S\) and by PCI it holds that \(f_{n+6}=4f_{n+3}+f_n\) for all \(n \in \mathbb{N}\)
  \end{enumerate}
\end{proof}  
\end{theorem1}

\newpage

\end{document}