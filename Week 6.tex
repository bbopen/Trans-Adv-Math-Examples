\documentclass[a4paper,11pt]{article}
\usepackage{amsmath,amsthm,amssymb}
\usepackage{mathtools}
\usepackage{mathrsfs}
\usepackage{setspace}
\DeclarePairedDelimiter\abs{\lvert}{\rvert}
\begin{document}
\newtheorem*{theorem1}{Theorem}
\newtheorem*{theorem2}{Theorem}
\newtheorem*{theorem3}{Theorem}
\newtheorem*{theorem4}{Theorem}
\newtheorem*{theorem5}{Theorem}
\newtheorem*{theorem6}{Theorem}
\newtheorem*{theorem7}{Theorem}
\newtheorem*{theorem8}{Theorem}
\title{MATH 393 Week 6 Assignment 2\textsuperscript{nd} Draft}
\author{Brett G. Bonner}
\date{February 24, 2014}
\maketitle
\linespread{1.5}
\doublespacing
\newcounter{ProblemCounter}
\newcounter{SubsectionCounter}[ProblemCounter]
\addtocounter{ProblemCounter}{1} % set them to some other numbers than 0
\addtocounter{SubsectionCounter}{10} % same
%

\section*{\S 2.3 Exercise \arabic{ProblemCounter}: Find the union and intersection of each of the following families or indexed
collections.}
\textbf{\arabic{ProblemCounter}.\alph{SubsectionCounter}}
For each \(n \in \mathbb{N}\) let \(M_n = \{\ldots, -3n, -2n, -n, 0, n, 2n, 3n, \ldots\}\) and let \(\mathscr{M} = \{M_n: n \in \mathbb{N}\}\) \\
Union:\\
If \(n=1\), \(M_n = \{\ldots, -3, -2, -1, 0, 1, 2, 3, \ldots\} = \mathbb{Z}\)\\
If \(n=2\), \(M_n = \{\ldots, -6, -4, -2, 0, 2, 4, 6, \ldots\}\)\\
\(\bigcup\limits_{n \in \mathbb{N}}M_n = \mathbb{Z}\)\\

\noindent Intersection:\\
\(\bigcap\limits_{n \in \mathbb{N}}M_n = \{0\}\) since $\ldots$\\
0 is an element of all sets \(M_n\) and \\
For any \(n \in \mathbb{N}\), \(n \notin M_{n+1}\) and \(-n \notin M_{n+1}\) (for any natural number, neither the number nor its additive inverse are elements of the subsequent set indexed n+1)


\newpage
\setcounter{SubsectionCounter}{4}
\setcounter{ProblemCounter}{3}
\section*{\S 2.3 Exercise \arabic{ProblemCounter}: Prove part (b) of Theorem 2.3.1: \(B \subseteq \bigcup\limits_{A\in\mathscr{A}}A.\)}
\begin{theorem1}
\(B \subseteq \bigcup\limits_{A\in\mathscr{A}}A.\)
\begin{proof}
  Let \(\mathscr{A}\) be a family of sets.\\
  Let \(B \in \mathscr{A}\)\\
  Let \(x\) be arbitrary, and let \(x \in B\).\\
  As \(x \in B\) and \(B \in \mathscr{A}\), by definition then \(x \in \bigcup\limits_{A\in\mathscr{A}}A\), then there exists an \(A \in \mathscr{A}\) such that \(x \in A\).\\
  Therefore, \(B \subseteq \bigcup\limits_{A\in\mathscr{A}}A\).\\
  Because we let \(B \in \mathscr{A}\) and \(x \in B\), we proved \(B \subseteq \bigcup\limits_{A\in\mathscr{A}}A\).
\end{proof}
\end{theorem1}

\newpage

\setcounter{ProblemCounter}{4}
\setcounter{SubsectionCounter}{1}
\section*{\S 2.3 Exercise \arabic{ProblemCounter}: Let the universe of discourse be the set \(\mathbb{R}\) of real numbers, and let \(\mathscr{A}\) be the
empty family of subsets of \(\mathbb{R}\).:}
\textbf{\arabic{ProblemCounter}.\alph{SubsectionCounter}}
Show that \(\bigcap\limits_{A\in\mathscr{A}}A=\mathbb{R}\)
\begin{theorem2}
\(\bigcap\limits_{A\in\mathscr{A}}A=\mathbb{R}\) if \(\mathscr{A}\) is the empty 
family of subsets of \(\mathbb{R}\)
   \begin{proof}
    Let the universe of discourse be the set \(\mathbb{R}\) of real numbers. Let 
    \(\mathscr{A}\) be the empty family of subset \(\mathbb{R}\).\\\\
    For equivalence, we must show that (i) \(\bigcap\limits_{A \in \mathscr{A}}A \subseteq \mathbb{R}\) and that (ii) \(\mathbb{R} \subseteq \bigcap\limits_{A \in 
    \mathscr{A}}A\).\\\\
    i) Let \(x\) be arbitrary and \(x \in \bigcap\limits_{A \in \mathscr{A}}A\). As \(\mathscr{A}\) 
    is empty, there is no \(A\) such that \(A \in \mathscr{A}\). Thus \(`x \in A\) for every \(A \in \mathscr{A}\)' holds for all \(x \in \mathscr{R}\).\\
    Thus if \(x \in \bigcap\limits_{A \in A}A, \text{ then } x \in \mathbb{R}\).\\
    Thus \(\bigcap\limits_{A \in \mathscr{A}}A \subseteq \mathbb{R}\).\\\\
    ii) Let \(x\) be an arbitrary element of \(\mathbb{R}\). Show that if \(x \in 
    \mathbb{R}\), then \(x \in \bigcap\limits_{A \in \mathscr{A}}A\).
    As \(\mathscr{A}\) is empty, then there is no \(A\) such that \(A \in \mathscr{A}\).
    Thus the statement, if \(A \in \mathscr{A}\) then \(x \in A\) is always 
    true, because the antecedent is always false.\\
    Therefore, for every \(A \in \mathscr{A}, x \in A\).\\
    Thus if \(x \in \mathscr{R}\), then \(x \in \bigcap\limits_{A \in \mathscr{A}}A\) 
    by definition.\\
    Thus \(\mathbb{R} \subseteq \bigcap\limits_{A \in \mathscr{A}}A\)
  \end{proof}
\end{theorem2}

\newpage
\section*{\S 2.3 Exercise \arabic{ProblemCounter}: Let the universe of discourse be the set \(\mathbb{R}\) of real numbers, and let \(\mathscr{A}\) be the
empty family of subsets of \(\mathbb{R}\).:}
\noindent \setcounter{SubsectionCounter}{2}
\textbf{\arabic{ProblemCounter}.\alph{SubsectionCounter}}
Show that \(\bigcup\limits_{A\in\mathscr{A}}A=\varnothing\)
\begin{theorem2}
\(\bigcup\limits_{A\in\mathscr{A}}A=\varnothing\)
  \begin{proof}
    Let the universe of discourse be the set \(\mathbb{R}\) of real numbers. Let 
    \(\mathscr{A}\) be the empty family of subset \(\mathbb{R}\).\\
    Suppose \(\bigcup\limits_{A\in\mathscr{A}}A\neq\varnothing\).\\
    Let \(x\) be arbitrary where \(x \in \bigcap\limits_{A \in \mathscr{A}}A\), 
    then \(x \in A\) for some \(A \in \mathscr{A}\).\\
    Since \(\mathscr{A}\) is empty, there is a contradiction as there is no \(A\) such that \(A \in 
    \mathscr{A}\). Therefore \(\bigcup\limits_{A\in\mathscr{A}}A=\varnothing\).
  \end{proof}
\end{theorem2}

\noindent \setcounter{SubsectionCounter}{3}
\textbf{\arabic{ProblemCounter}.\alph{SubsectionCounter}}
Conclude that \(\bigcap\limits_{A\in\mathscr{A}}A \subseteq \bigcup\limits_{A\in\mathscr{A}}A\) 
is false in this example
\begin{theorem2}
\(\bigcap\limits_{A\in\mathscr{A}}A \subseteq \bigcup\limits_{A\in\mathscr{A}}A\) 
is false if \(\mathscr{A}\) is an empty family of subset \(\mathbb{R}\)..
   \begin{proof}
    Let the universe of discourse be the set \(\mathbb{R}\) of real numbers and let \(\mathscr{A}\) be the empty family of subsets of \(\mathbb{R}\).\\
    As demonstrated by {(4.a)}, \(\bigcap\limits_{A\in\mathscr{A}}A=\mathbb{R}\).\\
    As demonstrated by {(4.b)}, 
    \(\bigcup\limits_{A\in\mathscr{A}}A=\varnothing\).\\
    By definition of a subset, \(\bigcap\limits_{A\in\mathscr{A}}A \subseteq \bigcup\limits_{A\in\mathscr{A}}A\)  
    if there exists an arbitrary \(x\) such that \(x \in \bigcap\limits_{A\in\mathscr{A}}A\) 
    and \(x \in \bigcup\limits_{A\in\mathscr{A}}A\).\\
    Let an arbitrary \(x \in x \in \bigcap\limits_{A\in\mathscr{A}}A\). 
    Therefore \(x \in \bigcup\limits_{A\in\mathscr{A}}A\), which is a 
    contradiction as \(\bigcup\limits_{A\in\mathscr{A}}A=\varnothing\).\\
    Therefore we proved \(\bigcap\limits_{A\in\mathscr{A}}A \subseteq \bigcup\limits_{A\in\mathscr{A}}A\) 
is false if \(\mathscr{A}\) is an empty family of subset \(\mathbb{R}\).
  \end{proof}
\end{theorem2}

\newpage

\setcounter{ProblemCounter}{5}
\setcounter{SubsectionCounter}{1}
\section*{\S 2.3 Exercise \arabic{ProblemCounter}:}
\textbf{\arabic{ProblemCounter}.\alph{SubsectionCounter}}
Prove parts (a) and (b) of Theorem 2.3.2\\
Theorem 2.3.2: Let \(\mathscr{A} = \{A_\alpha:a \in \Delta\}\) be an indexed collection of 
sets.\\
\textbf{(a)}: \(\bigcap\limits_{\alpha \in \Delta}A_\alpha\subseteq A_{\beta}\) for each \(\beta \in 
\Delta\).
\begin{theorem2}
\(\bigcap\limits_{\alpha \in \Delta}A_\alpha\subseteq A_{\beta}\) for each \(\beta \in 
\Delta\)
\begin{proof}
  Let \(\mathscr{A}\) be an indexed collection of sets, \(A_\beta \in \mathscr{A}\), and let \(\beta \in \Delta\).\\
  Suppose an arbitrary \(x \in \bigcap\limits_{\alpha \in \Delta}A_\alpha\). Then \(x \in A_\alpha\) 
  for each \(\alpha \in \Delta\).\\
  Since \(\beta \in \Delta\), therefore \(x \in A_\beta\).\\ 
  Thus \(\bigcap\limits_{\alpha \in \Delta}A_\alpha\subseteq A_{\beta}\) for each \(\beta \in 
  \Delta\).\\
\end{proof}
\end{theorem2}

\noindent \textbf{(b)}: \(A_{\beta} \subseteq \bigcup\limits_{\alpha \in \Delta}A_\alpha \) for each \(\beta \in 
\Delta\).
\begin{theorem2}
\(A_{\beta} \subseteq \bigcup\limits_{\alpha \in \Delta}A_\alpha \) for each \(\beta \in 
\Delta\).
\begin{proof}
  Let \(\mathscr{A}\) be an indexed collection of sets, \(\bigcup\limits_{\alpha \in \Delta}A_\alpha \in \mathscr{A}\), and let \(\beta \in \Delta\).\\
  Suppose an arbitrary \(x \in A_\beta\).\\
  If \(A_{\beta} \subseteq \bigcup\limits_{\alpha \in \Delta}A_\alpha \), then \(x \in A_\alpha\) 
  and there is exists an \(\alpha \in \Delta\).\\
  Since \(\beta \in \Delta\), \(x \in \bigcup\limits_{\alpha \in \Delta}A_\alpha\) and thus \(A_{\beta} \subseteq \bigcup\limits_{\alpha \in \Delta}A_\alpha \) for each \(\beta \in 
\Delta\)\\
\end{proof}
\end{theorem2}
\newpage
\noindent \setcounter{SubsectionCounter}{2}
\textbf{\arabic{ProblemCounter}.\alph{SubsectionCounter}}
Give a direct proof of part (d) of Theorem 2.3.2 that does not use part (c).\\
Theorem 2.3.2 (d): \(\left(\bigcup\limits_{\alpha \in \Delta}A_\alpha\right)^c = \bigcap\limits_{\alpha \in \Delta}A\limits_{\alpha}^c \)
\begin{theorem2}
  \(\left(\bigcup\limits_{\alpha \in \Delta}A_\alpha\right)^c = \bigcap\limits_{\alpha \in \Delta}A\limits_{\alpha}^c \)
  \begin{proof}
    \(\left(\bigcup\limits_{\alpha \in \Delta}A_\alpha\right)^c \)
    \begin{align*}
      &\text{iff } x \notin \bigcup\limits_{\alpha \in \Delta}A_\alpha\\
      &\text{iff it is not the case that there is an } \alpha \in \Delta, x \in A_\alpha\\
      &\text{iff for some } \beta \in \Delta, x \notin A_\beta\\
      &\text{iff for some } \beta \in \Delta, x \in A\limits_{\beta}^c\\
      &\text{iff } \bigcap\limits_{\alpha \in \Delta}A\limits_{\alpha}^c
    \end{align*}
    Therefore, \(\left(\bigcup\limits_{\alpha \in \Delta}A_\alpha\right)^c = \bigcap\limits_{\alpha \in \Delta}A\limits_{\alpha}^c \)
  \end{proof}
\end{theorem2}

\newpage

\setcounter{ProblemCounter}{16}
\setcounter{SubsectionCounter}{1}
\section*{\S 2.3 Exercise \arabic{ProblemCounter}: Suppose \(\mathscr{A}=\{A_i: i \in \mathbb{N}\}\) is a family of sets such that for all \(i, j \in \mathbb{N}\) if \(i \leq j\), then \(A_j \subseteq A_i\). (Such a family is called a \textit{decreasing nested family} of sets).}
\textbf{\arabic{ProblemCounter}.\alph{SubsectionCounter}}
Prove that for every \(k \in \mathbb{N}, \bigcap\limits_{i=1}^kA_i=A_k\).
\begin{theorem2}
For every \(k \in \mathbb{N}, \bigcap\limits_{i=1}^kA_i=A_k\)
  \begin{proof}
    Let \(k \in \mathbb{N}\) be arbitrary.\\
    For equivalence, we must show \(\bigcap\limits_{i=1}^kA_i \subseteq A_k\) and \(A_k \subseteq \bigcap\limits_{i=1}^kA_i\)\\
    \textbf{i) \(\bigcap\limits_{i=1}^kA_i \subseteq A_k\)}\\
    Let \(x \in \bigcap\limits_{i=1}^kA_i\) be arbitrary.\\
    By definition of intersection of a family, \(x \in A_i\) for all \(i = 1, 2, \ldots k\).\\
    Therefore \(x \in A_k\) and we showed \(\bigcap\limits_{i=1}^kA_i \subseteq 
    A_k\).\\
    \textbf{ii) \(A_k \subseteq \bigcap\limits_{i=1}^kA_i\)}\\
    Let \(x \in A_k\) be arbitrary.\\
    Let \(i \leq k\) be arbitrary.\\
    Because the family is nested, \(A_k \subseteq A_i\).\\
    Since \(x \in A_k\) and \(A_k \subseteq A_i\), we conclude \(x \in A_i\).\\
    Because \(i \leq k\) was arbitrary, \(x \in A_i\) for all \(i \leq k\).\\
    So \(x \in \bigcap\limits_{i=1}^kA_i\) and therefore \(A_k \subseteq \bigcap\limits_{i=1}^kA_i\).\\
    \textbf{Conclusion:}
    Since we proved \(\bigcap\limits_{i=1}^kA_i \subseteq A_k\) and \(A_k \subseteq \bigcap\limits_{i=1}^kA_i\), \(\bigcap\limits_{i=1}^kA_i=A_k\).\\
    Because we let \(k \in \mathbb{N}\) be arbitrary, we proved for all \(k \in 
    \mathbb{N}\), \(\bigcap\limits_{i=1}^kA_i=A_k\)
  \end{proof}
\end{theorem2}

\end{document}