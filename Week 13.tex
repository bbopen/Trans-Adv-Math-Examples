\documentclass[a4paper,11pt]{article}
\usepackage{amsmath,amsthm,amssymb}
\usepackage{mathtools}
\usepackage{mathrsfs}
\usepackage{setspace}
\usepackage{enumerate}
\usepackage{caption}
\usepackage{pgfplots}
\DeclarePairedDelimiter\abs{\lvert}{\rvert}
\PassOptionsToPackage{usenames,dvipsnames,svgnames}{xcolor}  
\usepackage{tikz}
\usepackage{cancel}
\usetikzlibrary{arrows,positioning,automata,fit,shapes,calc,backgrounds}
\usepackage{framed}
\newcommand{\cupdot}{\mathbin{\mathaccent\cdot\cup}}
\begin{document}
\newtheorem*{theorem1}{Theorem}
\newtheorem*{theorem2}{Theorem}
\newtheorem*{theorem3}{Theorem}
\newtheorem*{theorem4}{Theorem}
\newtheorem*{theorem5}{Theorem}
\newtheorem*{theorem6}{Theorem}
\newtheorem*{theorem7}{Theorem}
\newtheorem*{theorem8}{Theorem}
\newtheorem*{theorem9}{True/False?}
\title{MATH 393 Week 12 Assignment 2\textsuperscript{nd} Draft}
\author{Brett Bonner}
\date{April 20, 2014}
\maketitle
\doublespacing
\newcounter{ProblemCounter}
\newcounter{SubsectionCounter}[ProblemCounter]
\setcounter{ProblemCounter}{2}
\section*{\S 4.5 Exercise \arabic{ProblemCounter}: Let \(f{(x)}=x^{2} + 1\). Find:}
\begin{figure}[htbp!]
\centering
\includegraphics[width=1.0\textwidth]{plot.png}
\label{fig:plot}
\end{figure}
\setcounter{SubsectionCounter}{1}
\noindent\textbf{\arabic{ProblemCounter}.\alph{SubsectionCounter}}
\(f{([1,3])}\) = \([2,10]\) because \({(1)}^2 + 1 = 2 \text{ and } {(3)}^2 + 1 = 10.\)\\
\addtocounter{SubsectionCounter}{2}
\textbf{\arabic{ProblemCounter}.\alph{SubsectionCounter}}
\(f^{-1}{([-1,1])} = \{0\}\) by inspection, because \(f{(x)}=1\) only at \(x=0\)\\
\addtocounter{SubsectionCounter}{2}
\textbf{\arabic{ProblemCounter}.\alph{SubsectionCounter}}
\(f^{-1}{({(5,{10]}})} = {[-3,-2)} \cup {(2,3]}\) by inspection, as \(-3 \leq x < -2 \) and \(2 < x \leq 3\) where \(5 < y \leq 10 
\).
\newpage
\setcounter{ProblemCounter}{7}
\section*{\S 4.5 Exercise \arabic{ProblemCounter}: Prove parts {(b)} and {(d)} of Theorem 4.5.1}
\setcounter{SubsectionCounter}{2}
\textbf{\arabic{ProblemCounter}.\alph{SubsectionCounter}}
\begin{theorem1}
  Let \(F:A \rightarrow B\), \(C\) and \(D\) be subsets of \(A\), then \(f{(C \cup D)}=f{(C)} \cup f{(D)}\)
  \begin{proof}
    \begin{align*}
      &f{(C \cup D)}=f{(C)} \cup f{(D)}\\
      &\Leftrightarrow f{(C \cup D)} \subseteq f{(C)} \cup f{(D)} \text{ and } f{(C)} \cup f{(D)} \subseteq f{(C \cup 
      D)}\\
      &\Leftrightarrow x \in C \cup D \text{ such that } f{(x)} = y\\
      &\Leftrightarrow y \in f{(C \cup D)} \text{ and } {(x \in C \text{ or } x \in D)}\\
      &\Leftrightarrow y \in f{(C)} \cup f{(D)}\\
      &\Leftrightarrow y \in f{(C)} \text{ or } y \in f{(D)}\\
    \end{align*}
\end{proof}
\end{theorem1}
\newpage
\setcounter{SubsectionCounter}{4}
\noindent\textbf{\arabic{ProblemCounter}.\alph{SubsectionCounter}}
\begin{theorem1}
  Let \(F:A \rightarrow B\), \(C\) and \(D\) be subsets of \(A\), and \(E\) and \(F\) be subsets of \(B\), then \(f^{-1}{(E \cup F)} = f^{-1}{(E)} \cup f^{-1}{(F)}\)
  \begin{proof}
    \begin{align*}
      &f^{-1}{(E \cup F)} = f^{-1}{(E)} \cup f^{-1}{(F)}\\
      &\Leftrightarrow f^{-1}{(E \cup F)} \subseteq f^{-1}{(E)} \cup f^{-1}{(F)} \text{ and } f^{-1}{(E)} \cup f^{-1}{(F)} \subseteq f^{-1}{(E \cup 
    F)}\\
      &\Leftrightarrow x \in f^{-1}{(E \cup F)} \text{ and } x \in f^{-1}{(E)} \cup f^{-1}{(F)}\\
      &\Leftrightarrow f{(x)} = y \text{ for some } y \in E \cup F\\
      &\Leftrightarrow f{(x)} \in E \cup F\\
      &\Leftrightarrow f{(x)} \in E \text{ or } f{(x)} \in F\\
      &\Leftrightarrow x \in f^{-1}{(E)} \text{ or } x \in f^{-1}{(F)}
    \end{align*}
  \end{proof}
\end{theorem1}
\newpage
\setcounter{ProblemCounter}{1}
\section*{\S 5.1 Exercise \arabic{ProblemCounter}: Prove Theorem 5.1.1. That is, show that the relation \(\approx\) is reflexive, symmetric, and transitive on the class of all sets.}
\setcounter{SubsectionCounter}{2}
\begin{theorem1}
  The equivalence of sets is an equivalence relation on the class of all sets.
  \begin{proof}
    \begin{enumerate}[(i)]
      \item \(\approx\) is reflexive.\\
      Let \(A\) be a set.\\
      We show there exists a function \(f:A \xrightarrow[onto]{1-1} A\).\\
      Let \(f:A \rightarrow A\) be given by \(f{(x)}=x\).\\
      \(f{(x)}=x\) is one-to-one, because \(f{(x)}=f{(y)} \text{ yields } x = y\).\\
      \(f\) is onto \(A\), because for every \(x \in A, f{(x)}=x\).\\
      Therefore, \(A \approx A\).\\
      Since we let \(A\) be a set, we showed that the relation \(\approx\) is 
      reflexive.
      \item \(\approx\) is symmetric.\\
      Let \(A\), \(B\) be sets. We want to show if \(A \approx B\) then \(B \approx 
      A\).\\
      Let \(A \approx B\). Then \(f:A \xrightarrow[onto]{1-1} B\).\\
      By Corollary 4.4.3, since \(f\) is bijection, \(f^{-1}\) is bijection.\\
      Then let \(f^{-1}\) be \(g:B \xrightarrow[onto]{1-1}A\).\\
      Therefore \(B \approx A\).\\
      Since we let \(A\) and \(B\) be sets, we showed if \(A \approx B\) then \(B \approx 
      A\) and therefore the relation \(\approx\) is symmetric.
      \newpage
      \item \(\approx\) is transitive.\\
      Let \(A\), \(B\), and \(C\) be sets.\\
      We want to show if \(A \approx B\) and \(B \approx C\), then \(A \approx 
      C\).\\
      Assume \(A \approx B\) and \(B \approx C\).\\
      Since \(A \approx B\), \(f:A \xrightarrow[1-1]{onto} B\).\\
      Since \(B \approx C\), \(g:B \xrightarrow[1-1]{onto} C\).\\
      Then \(g \circ f:A \xrightarrow[1-1]{onto} C \) by Theorem 4.4.1.\\
      Therefore if \(A \approx B\) and \(B \approx C\), then \(A \approx C\).\\
      Thus, the relation \(\approx\) is transitive.
    \end{enumerate}
    \noindent By {(i)}, {(ii)}, and {(iii)} we showed that the equivalence of sets is an equivalence relation on the class of all sets. That is, the relation \(\approx\) is 
    reflexive, symmetrix, and transitive on the class of all sets.
  \end{proof}
\end{theorem1}
\newpage
\setcounter{ProblemCounter}{4}
\section*{\S 5.1 Exercise \arabic{ProblemCounter}: Complete the proof of Lemma 5.1.2{(b)} by showing that if \(h:A \rightarrow C\) and \(g:B \rightarrow D\) are one-to-one correspondences, the \(f:A \times B \rightarrow C \times D\) given by \(f{(a,b)} = {(h{(a)},g{(b)})}\) is a one-to-one correspondence.}
\setcounter{SubsectionCounter}{2}
\textbf{\arabic{ProblemCounter}.\alph{SubsectionCounter}}
\begin{theorem1}
  Suppose \(A,B,C,\) and \(D\) are sets with \(A \approx C\) and \(B \approx 
  D\). \(A \times B \approx C \times D\).
  \begin{proof}
    Let \(f:A \times B \rightarrow C \times D\) be given by \(f{(a,b)} = 
    {(h{(a)},g{(b)})}\), where \(h:A \xrightarrow[onto]{1-1} C\) and \(g:B \xrightarrow[onto]{1-1} 
    D\) from Lemma 5.1.2{(a)}.
    \begin{enumerate}[(i)]
      \item Show \(f: A \times B \xrightarrow{onto} C \times D\).\\
      Let \({(c,d)} \in C \times D\) be arbitrary.\\
      Since \(h:A \xrightarrow[onto]{1-1} C\), there exists \(a \in A\) such 
      that \(h{(a)} = c\).\\
      Since \(g:B \xrightarrow[onto]{1-1} D\), there exists \(b \in B\) such 
      that \(g{(b)} = d\).\\
      Thus \({(c,d)}={(h{(a)},g{(b)})}\).\\
      Then there exists an \({(a,b)} \in {(A \times B)}\) such that \(f{(a,b)} = 
      {(c,d)}\).\\
      Since we showed for all arbitrary \({(c,d)} \in C \times D\) that there exists 
      \({(a,b)} \in {(A \times B)}\) such that \(f{(a,b)} = {(c,d)}\), \\
      then we conclude \(f:A \times B \xrightarrow{onto} C \times D\).
      \newpage
      \item Show \(f: A \times B \xrightarrow{1-1} C \times D\).
      Let \(f{(x,y)} = f{(m,n)}\).\\
      Then \(f{(x,y)} = {(h{(x)},g{(y)})} = {(h{(m)}, g{(n)}}) = f{(m,n)}\).\\
      Then \(h{(x)}=h{(m)}\) and \(g{(y)} = g{(n)}\).\\
      Since \(h:A \xrightarrow[onto]{1-1} C\) and \(g:B \xrightarrow[onto]{1-1} 
    D\), then \(x=m\) and \(y=m\).\\
    Then \({(x,y)}={(m,n)}\).\\
    Thus \(f: A \times B \xrightarrow{1-1} C \times D\).
    \end{enumerate}
    From {(i)} and {(ii)}, we have shown \(f:A \times B \xrightarrow[onto]{1-1} C \times D\).\\ 
    Thus proving \(A \times B \approx C \times D\).
  \end{proof}
\end{theorem1}
\newpage
\setcounter{ProblemCounter}{7}
\section*{\S 5.1 Exercise \arabic{ProblemCounter}: Using the methods of this section, prove that if \(A\) and \(B\) are finite sets, then \( \overline{\overline{A \cup B}} = \overline{\overline{A}} + \overline{\overline{B}} - \overline{\overline{A \cap B}}\). This fact is a restatement of Theorem 2.6.1}
\begin{theorem1}
if \(A\) and \(B\) are finite sets, then \( \overline{\overline{A \cup B}} = \overline{\overline{A}} + \overline{\overline{B}} - \overline{\overline{A \cap B}}\).
  \begin{proof}
Suppose \(A\) and \(B\) are finite sets.\\
We want to show \( \overline{\overline{A \cup B}} = \overline{\overline{A}} + \overline{\overline{B}} - \overline{\overline{A \cap 
B}}\).\\
Recall 5.1.7a: If \(A\), \(B\) are finite and disjoint, then \(\overline{\overline{A \cup B}} = \overline{\overline{A}} + 
\overline{\overline{B}}\).\\
We observe that \(B={(B - A)} \cup {(A \cap B)}\) and \({(B - A)} \cap {(A \cap B)} = 
\varnothing\).\\
So \(\overline{\overline{B}} = \overline{\overline{{(B -A)}}} + \overline{\overline{(A \cap 
B)}}\).\\
So \(\overline{\overline{B-A}}=\overline{\overline{B}} - \overline{\overline{A \cap 
B}}\).\\
\(A \cup B = A \cupdot {(B - A)}\).\\
\(\overline{\overline{A \cup B}} = \overline{\overline{A}} + \overline{\overline{B-A}}.\)\\
We therefore conclude if \(A\) and \(B\) are finite sets, \(\overline{\overline{A \cup B}} = \overline{\overline{A}} + \overline{\overline{B}} - \overline{\overline{A \cap 
B}}.\)\\
  \end{proof}
\end{theorem1}
\newpage
\setcounter{ProblemCounter}{8}
\section*{\S 5.1 Exercise \arabic{ProblemCounter}: Prove part {(c)} of Theorem 5.1.7}
\setcounter{SubsectionCounter}{3}
\textbf{\arabic{ProblemCounter}.\alph{SubsectionCounter}}
\begin{theorem1}
If \(A_{1},A_{2},\ldots,A_{n}\) are finite sets, then \(\bigcup\limits_{i=1}^{n}A_{i}\) 
is finite.
  \begin{proof}
  Let \(A_{1},A_{2},\ldots,A_{n},A_{n+1}\) be finite sets for some \(n \in 
  \mathbb{N}\).\\
  Let \(S = \{n \in \mathbb{N}: \bigcup\limits_{i=1}^{n}A_{i} \text{ is 
    finite}\}\).
    \begin{enumerate}[(i)]
      \item Base case: show \(\bigcup\limits_{i=1}^{n}A_{i}\) is finite for 
      \(n=1\).\\
      \(\bigcup\limits_{i=1}^{1}A_{i} = A_1\).\\
      By our hypothesis, \(A_{1}\) is finite.\\
      Therefore \(1 \in S\).
      \item Inductive Hypothesis: for all \(n \in \mathbb{N}\), if \(n \in S\), 
      then \(n+1 \in S\).\\
      Assume \(n \in \mathbb{N}\) and \(n \in S\), that is assume \(\bigcup\limits_{i=1}^{n}A_{i}\) 
      is finite.\\
      We want to show \(\bigcup\limits_{i=1}^{n+1}A_{i}\) is finite.\\
      Let \(A_{1},A_{2},\ldots,A_{n},A_{n+1}\) be finite sets.\\
      \(\bigcup\limits_{i=1}^{n+1}A_{i} = \bigcup\limits_{i=1}^{n}A_{i} \cup 
      A_{n+1}\).\\
      Since \(A_{n+1}\) is finite and \(\bigcup\limits_{i=1}^{n}A_{i}\) is 
      finite, by Theorem 5.1.7b, \(\bigcup\limits_{i=1}^{n}A_{i} \cup 
      A_{n+1}\) is finite.\\
      Therefore \(\bigcup\limits_{i=1}^{n+1}A_{i}\) is finite.
      \item We showed by PMI that if \(A_{1},A_{2},\ldots,A_{n}\) are finite sets for some \(n \in 
      \mathbb{N}\), then \(\bigcup\limits_{i=1}^{n}A_{i}\) 
is finite.
    \end{enumerate}
  \end{proof}
\end{theorem1}
\newpage
\setcounter{ProblemCounter}{13}
\section*{\S 5.1 Exercise \arabic{ProblemCounter}: Prove that if \(r > 1\) and \(x \in \mathbb{N}_{r}\), then \(\mathbb{N}_{r}-\{x\} \approx \mathbb{N}_{r-1}\) (Lemma 5.1.8)}
\setcounter{SubsectionCounter}{3}
\textbf{\arabic{ProblemCounter}.\alph{SubsectionCounter}}
\begin{theorem1}
Prove that if \(r > 1\) and \(x \in \mathbb{N}_{r}\), then \(\mathbb{N}_{r}-\{x\} \approx \mathbb{N}_{r-1}\)
  \begin{proof}
  Let \(r > 1\) be a natural number and let some \(x \in \mathbb{N}_r\).
  \begin{enumerate}[(i)]
    \item Case 1: \(x = r\)\\
    Then \(\mathbb{N}_r - \{x\} = \mathbb{N}_{r-1}\), and clearly \(\mathbb{N}_{r-1} \approx \mathbb{N}_{r-1}\)
    \item Case 2: \(x < r\)\\
    Let \(A = \mathbb{N}_{x-1}\) and \(B = \mathbb{N}_{x-1}\).\\
    Let \(C = \{x+1,x+2,\ldots r\}\) and \(D = \{x, x+1, \ldots r-1\}\).\\
    Let \(n \in \mathbb{N}\) be arbitrary.\\
    Let \(h: A \rightarrow B\) be given by 
    \(h{(n)}=n\).\\
    Let \(g: B \rightarrow C\) be given by 
    \(g{(n)}=n-1\).\\
    \(h: A \xrightarrow[onto]{1-1} B\) and \(g: C \xrightarrow[onto]{1-1} D\).\\ 
    Since \(A \cap C = \varnothing\), by Theorem 4.3.5, \(h \cup g\) is a function and onto \(B \cup D\).\\
    \(B \cup D = \varnothing\), so \(h \cup g\) is one-to-one.\\
    Then \(h \cup g: N_{r} - \{x\} \xrightarrow[onto]{1-1} \mathbb{N}_{r-1}\).\\
    Therefore \(\mathbb{N}_{r} - \{x\} \approx \mathbb{N}_{r-1}\).
  \end{enumerate}
  We showed that if \(r > 1\) and \(x \in \mathbb{N}_r\), then \(\mathbb{N}_{r} - \{x\} \approx 
  \mathbb{N}_{r-1}\).
  \end{proof}
\end{theorem1}
\end{document}