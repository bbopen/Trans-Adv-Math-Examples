\documentclass[a4paper,11pt]{article}
\usepackage{amsmath,amsthm,amssymb}
\usepackage{mathtools}
\usepackage{mathrsfs}
\usepackage{setspace}
\usepackage{enumerate}
\usepackage{caption}
\usepackage{pgfplots}
\DeclarePairedDelimiter\abs{\lvert}{\rvert}
\PassOptionsToPackage{usenames,dvipsnames,svgnames}{xcolor}  
\usepackage{tikz}
\usepackage{cancel}
\usetikzlibrary{arrows,positioning,automata,fit,shapes,calc,backgrounds}
\usepackage{framed}
\newcommand{\cupdot}{\mathbin{\mathaccent\cdot\cup}}
\begin{document}
\newtheorem*{theorem1}{Theorem}
\newtheorem*{theorem2}{Theorem}
\newtheorem*{theorem3}{Theorem}
\newtheorem*{theorem4}{Theorem}
\newtheorem*{theorem5}{Theorem}
\newtheorem*{theorem6}{Theorem}
\newtheorem*{theorem7}{Theorem}
\newtheorem*{theorem8}{Theorem}
\newtheorem*{theorem9}{True/False?}
\title{MATH 393 Week 14 Assignment 1\textsuperscript{st} Draft}
\author{Brett Bonner}
\date{April 25, 2014}
\maketitle
\doublespacing
\newcounter{ProblemCounter}
\newcounter{SubsectionCounter}[ProblemCounter]
\setcounter{ProblemCounter}{3}
\section*{\S 5.2 Exercise \arabic{ProblemCounter} Prove that the following sets are denumerable}
\setcounter{SubsectionCounter}{2}
\textbf{\arabic{ProblemCounter}.\alph{SubsectionCounter}} \(3\mathbb{N}\), the 
positive integer of multiples of \(3\).\\
Let \(S = \{3, 6, 9, \ldots\}\)\\
Let \(f:\mathbb{N} \rightarrow S\) be defined by the function \(f{(x)} = 3x\)
    \begin{enumerate}[(i)]
      \item Show that \(f:\mathbb{N} \xrightarrow{1-1} S\).\\
      Let arbitrary \(x,y \in \mathbb{N}\). Suppose \(f{(x)}=f{(y)}\). Then \(3x = 3y\) 
      implies \(x = y\). Therefore \(f:\mathbb{N} \xrightarrow{1-1} S\)
      \item Show that \(f:\mathbb{N} \xrightarrow{onto} S\).\\
     Let an arbitrary \(w \in S\). Let \(t \in \mathbb{N}\) such that \(t = \frac{w}{3}\).\\
     Then \(f{(t)} = \frac{3w}{3} = w\). Therefore \(f:\mathbb{N} \xrightarrow{onto} S\).
    \end{enumerate}
    Because \(f: \mathbb{N} \xrightarrow[onto]{1-1} S\), we conclude \(\mathbb{N} \approx 3\mathbb{N}\) and therefore \(3\mathbb{N}\) is 
    denumerable.\\
\newpage
\setcounter{SubsectionCounter}{5}
\noindent\textbf{\arabic{ProblemCounter}.\alph{SubsectionCounter}} \(\{x:x \in \mathbb{Z} \text{ and } x < -12\}\).\\
Let \(S = \{-13,-14,-15,\ldots\}\)\\
Let \(f:\mathbb{N} \rightarrow S\) be defined by the function \(f{(x)} = -x - 12\)
    \begin{enumerate}[(i)]
      \item Show that \(f:\mathbb{N} \xrightarrow{1-1} S\).\\
      Let \(x,y \in \mathbb{N}\). Suppose \(f{(x)}=f{(y)}\).\\
      Then \(-x - 12 = -y - 12\) implies \(x = y\). Therefore \(f:\mathbb{N} \xrightarrow{1-1} S\)
      \item Show that \(f:\mathbb{N} \xrightarrow{onto} S\).\\
     Let an arbitrary \(w \in S\). Let \(t \in \mathbb{N}\) such that \(t = -w - 12\).\\
     Then \(f{(t)} = -{(-w - 12)} - 12 = w + 12 - 12 = w\). Therefore \(f:\mathbb{N} \xrightarrow{onto} S\).
    \end{enumerate}
    Because \(f: \mathbb{N} \xrightarrow[onto]{1-1} S\), we conclude \(\mathbb{N} \approx S\).\\
    Therefore \(\{x:x \in \mathbb{Z} \text{ and } x < -12\}\) is 
    denumerable.\\
\newpage
\setcounter{ProblemCounter}{5}
\section*{\S 5.2 Exercise \arabic{ProblemCounter} State whether each of the following is true or false.}
\setcounter{SubsectionCounter}{1}
\textbf{\arabic{ProblemCounter}.\alph{SubsectionCounter}} If a set \(A\) is 
countable, then \(A\) is infinite.\\
False. By definition, a set is countable iff the set is finite or denumerable. A counterexample is the finite set \(A = \{1,2,3\}\).\\
\addtocounter{SubsectionCounter}{1}
\textbf{\arabic{ProblemCounter}.\alph{SubsectionCounter}} If a set \(A\) is 
denumerable, then \(A\) is countable.\\
True. By definition, a set is countable iff the set is finite or denumerable.\\
\addtocounter{SubsectionCounter}{1}
\textbf{\arabic{ProblemCounter}.\alph{SubsectionCounter}} If a set \(A\) is 
finite, then \(A\) is denumerable.\\
False. A counterexample is the finite set \(A = \{1,2,3\}\), which is not denumerable, since a function \(f:\mathbb{N} \rightarrow \{1,2,3\}\) is not one-to-one\\
\addtocounter{SubsectionCounter}{1}
\textbf{\arabic{ProblemCounter}.\alph{SubsectionCounter}} If a set \(A\) is 
uncountable, then \(A\) is not denumerable.\\
True. By definition, a set is uncountable iff it is not countable. A set is 
countable iff it is finite or denumerable. Therefore if a set is not finite and not denumerable, then the set is uncountable.\\
\addtocounter{SubsectionCounter}{1}
\textbf{\arabic{ProblemCounter}.\alph{SubsectionCounter}} If a set \(A\) is 
uncountable, then \(A\) is not finite.\\
True. By similar reasoning of 5.d. By definition, a set is uncountable iff it is not countable. A set is 
countable iff it is finite or denumerable. Therefore if a set is not finite and not denumerable, then the set is uncountable.\\
\addtocounter{SubsectionCounter}{1}
\textbf{\arabic{ProblemCounter}.\alph{SubsectionCounter}} If a set \(A\) is 
not denumerable, then \(A\) is uncountable.\\
False. A set is countable iff it is finite or denumerable. A set may be finite and not denumerable, but remains countable.
\newpage
\setcounter{ProblemCounter}{7}
\section*{\S 5.2 Exercise \arabic{ProblemCounter} Which sets have cardinal number \(\aleph_{0}\)? \(\mathbf{ c }\)?}
\setcounter{SubsectionCounter}{2}
\textbf{\arabic{ProblemCounter}.\alph{SubsectionCounter}} \({(5,\infty)}\): \(\mathbf{c}\)
\begin{theorem1}
We consider \({(5,\infty)}\) has cardinal number \(\mathbf{c}\).\\
That is, \({(5,\infty)} \approx {(0,1)}\).
\begin{proof}
Let \(S = \{x \in \mathbb{R}: x > 5\}\).\\
Let \(f: {(0,1)} \rightarrow {(5,\infty)}\) be defined as a function \(f{(x)}=\tan{(\frac{\pi x}{2})} + 5\).
\begin{enumerate}[(i)]
  \item Show that \(f:{(0,1)} \xrightarrow{1-1} S\).\\
  Assume \(f{(x)} = 
  f{(y)}\).\\
  Then \(\tan{(\frac{\pi x}{2})} + 5 = \tan{(\frac{\pi y}{2})} + 5\).\\
  \(\arctan{[\tan{(\frac{\pi x}{2})} = \tan{(\frac{\pi y}{2})}]}\).\\
  \(\frac{\pi x}{2}=\frac{\pi y}{2}\).\\
  \(x=y\).\\
  Therefore \(f:{(0,1)} \xrightarrow{1-1} S\)
  \item Show that \(f:{(0,1)} \xrightarrow{onto} S\).\\
  Let \(w \in S\) be arbitrary. Let \(t \in {(0,1)}\) such that \(t = \frac{2 \tan^{-1}{(w - 5)}}{\pi}
  \).\\
  Then \(f{(t)} = \tan{(\frac{2\pi\tan^{-1}{(w-5)}}{2\pi})}+5 = w\).\\
  Therefore \(f:{(0,1)} \xrightarrow{onto} S\).
\end{enumerate}
Since \(f:{(0,1)} \xrightarrow[onto]{1-1} S\), \({(0,1)} \approx {(5,\infty)}\), thus \({(5,\infty)}\) has cardinal number \(\mathbf{c}\).
\end{proof}
\end{theorem1}
\newpage
\setcounter{SubsectionCounter}{4}
\noindent\textbf{\arabic{ProblemCounter}.\alph{SubsectionCounter}} \(\{2^{x}: x \in 
\mathbb{N}\}\): \(\aleph_{0}\)
\begin{theorem1}
We consider \(\{2^{x}: x \in 
\mathbb{N}\}\) has cardinal number \(\aleph_{0}\). That is, \(\{2^{x}: x \in 
\mathbb{N}\}\) is dunerable.
\begin{proof}
Let \(S = \{2,4,8,16 \ldots\}\).\\
Let \(f: \mathbb{N} \rightarrow S\) be defined as a function \(f{(x)}=2^{x}\).
\begin{enumerate}[(i)]
  \item Show that \(f:\mathbb{N} \xrightarrow{1-1} S\).\\
  Assume \(f{(x)} = 
  f{(y)}\).\\
  Then \(2^{x} = 2^{y}\).\\ \(\log_{2}2^{x} = \log_{2}2^{y}\).\\ \(x=y\).\\
  Therefore \(f:\mathbb{N} \xrightarrow{1-1} S\)
  \item Show that \(f:\mathbb{N} \xrightarrow{onto} S\).\\
  Let \(w \in S\) be arbitrary. Let \(t \in \mathbb{N}\) such that \(t = 
  \log_2{w}\).\\
  Then \(f{(t)} = 2^{\log_{2}w} = w\).\\
  Therefore \(f:\mathbb{N} \xrightarrow{onto} S\).
\end{enumerate}
Since \(f:\mathbb{N} \xrightarrow[onto]{1-1} S\), \(S\) is denumerable and 
therefore \(\{2^{x}: x \in 
\mathbb{N}\}\) has cardinal number \(\aleph_{0}\).
\end{proof}
\end{theorem1}
\end{document}