\documentclass[a4paper,11pt]{article}
\usepackage{amsmath,amsthm,amssymb}
\usepackage{mathtools}
\usepackage{mathrsfs}
\usepackage{setspace}
\usepackage{enumerate}
\usepackage{caption}
\usepackage{pgfplots}
\DeclarePairedDelimiter\abs{\lvert}{\rvert}
\PassOptionsToPackage{usenames,dvipsnames,svgnames}{xcolor}  
\usepackage{tikz}
\usepackage{cancel}
\usetikzlibrary{arrows,positioning,automata,fit,shapes,calc,backgrounds}
\usepackage{framed}

\begin{document}
\newtheorem*{theorem1}{Theorem}
\newtheorem*{theorem2}{Theorem}
\newtheorem*{theorem3}{Theorem}
\newtheorem*{theorem4}{Theorem}
\newtheorem*{theorem5}{Theorem}
\newtheorem*{theorem6}{Theorem}
\newtheorem*{theorem7}{Theorem}
\newtheorem*{theorem8}{Theorem}
\newtheorem*{theorem9}{True/False?}
\title{MATH 393 Week 12 Assignment 2\textsuperscript{nd} Draft}
\author{Brett Bonner}
\date{April 11, 2014}
\maketitle
\doublespacing
\newcounter{ProblemCounter}
\newcounter{SubsectionCounter}[ProblemCounter]
\newpage
\setcounter{ProblemCounter}{5}
\section*{\S 4.3 Exercise \arabic{ProblemCounter}: Prove that if \(f:A \xrightarrow{onto} B, g:B \xrightarrow{onto} C,\) then \(g \circ f: A \xrightarrow{onto} C\) (Theorem 4.3.1):}
\setcounter{SubsectionCounter}{1}
\begin{theorem1}
  If \(f:A\xrightarrow{onto}B\) and \(g:B\xrightarrow{onto}C\), then \(g \circ f: A\xrightarrow{onto} 
  C\).
  \begin{proof}
Suppose \(f:A\xrightarrow{onto}B\),  \(g:B\xrightarrow{onto}C\). Let \(w \in C\).\\
Since \(g:B\xrightarrow{onto}C\), there exits \(t \in B\) be such that \(g{(t)}=w\).\\
Since \(f:A \xrightarrow{onto} B\), there exists \(x \in A\) be such that \(f{(x)}=t\).\\
As \(f{(x)}=t\), \(w=g{(t)}=g{(f{(x)})}={(g \circ f)}{(x)}\).\\
Because we let an arbitrary \(w \in C\), and \(t \in B\) such that \(g{(t)}=w\)
and we let an \(x \in A\) such that \(f{(x)}=t\), we showed that if \(f:A\xrightarrow{onto}B\) 
and \(g:B\xrightarrow{onto}C\), then \(g \circ f: A\xrightarrow{onto} 
  C\).
  \end{proof}
\end{theorem1}
\newpage
\addtocounter{ProblemCounter}{1}
\section*{\S 4.3 Exercise \arabic{ProblemCounter}: Prove that if \(f:A \rightarrow B, g:B \rightarrow C,\) and \(g \circ f: A \xrightarrow{1-1} C\), then \(f:A \xrightarrow{1-1} B\) (Theorem 4.3.4):}
\setcounter{SubsectionCounter}{1}
\begin{theorem1}
if \(f:A \rightarrow B, g:B \rightarrow C,\) and \(g \circ f: A \xrightarrow{1-1} C\), then \(f:A \xrightarrow{1-1} 
B\). That is, if the composite of two functions is one-to-one, then the first 
function applied must be one-to-one.
\begin{proof}Suppose \(f: A \rightarrow B\) and \(g:B \rightarrow C\) are functions, and that \(g \circ f: A \xrightarrow{1-1} C\).\\
Let \(x,y \in A\) and \(f{(x)} = f{(y)}\).\\
Then since \(g\) is a function, \(g{(f{(x)})} = g{(f{(y)})}\).\\
Therefore \({(g \circ f)}{(x)} = {(g \circ f)}{(y)}\)\\
Since \(g \circ f: A \xrightarrow{1-1} C\), by definition \({(g \circ f)}{(x)} = {(g \circ f)}{(y)}\) and 
\(x=y\).\\
Thus, because we let \(x,y \in A\) and \(f{(x)}=f{(y)}\) and showed \(x=y\),\\
We can conclude that \(f:A \xrightarrow{1-1} B\).
  \end{proof}
\end{theorem1}
\newpage
\addtocounter{ProblemCounter}{1}
\section*{\S 4.3 Exercise \arabic{ProblemCounter}: Prove parts {(a)} and {(b)} of Theorem 4.3.5:}
\setcounter{SubsectionCounter}{1}
\textbf{\arabic{ProblemCounter}.\alph{SubsectionCounter}}
\begin{theorem1}
  A restriction of a one-to-one function is one-to-one.
  \begin{proof}
Let \(D \subseteq A\), \(f:A \xrightarrow{1-1} B\).\\
To show \(f|_{D} A \xrightarrow{1-1} B\), let \(x,y \in D\) and \(f|_{D}{(x)} = f|_{D}{(y)}\).\\
Then \(f|_{D}{(x)} = f{(x)}, \) and \(f|_{D}{(y)} = f{(y)}\).\\
Therefore \(f{(x)}=f{(y)}\).\\
As \(f:A \xrightarrow{1-1} B\) and since  \(f{(x)}=f{(y)}\), then \(x=y\).\\
Since we assumed \(x,y \in D\), \(D \subseteq A\), \(f: A \xrightarrow{1-1} B\), and  \(f|_{D}{(x)} = f|_{D}{(y)}\)\\ we showed 
\(x=y\) to conclude \(f|_{D} A \xrightarrow{1-1} B\).\\
  \end{proof}
  Biconditional attempt: please give feedback but do not grade.
\end{theorem1}
\newpage
\addtocounter{SubsectionCounter}{1}
\noindent\textbf{\arabic{ProblemCounter}.\alph{SubsectionCounter}}
\begin{theorem1}
  If \(h:A \xrightarrow{onto} C, g:B\xrightarrow{onto}D, \text{ and } A \cap B = \varnothing\), 
  then \(h \cup g: A \cup B \xrightarrow{onto} C \cup D\).
  \begin{proof}
Let \(h: A \xrightarrow{onto} C\), \(g: B \xrightarrow{onto} D\), and \(A \cap B = 
\varnothing\).\\
\(h \cup g: A \cup B \xrightarrow{onto} C \cup D\) we must show \(Rng{(h \cup g)} = C \cup D\).\\
That is, show \(C \cup D \subseteq Rng{(h \cup g)}\), since we know \(Rng{(h \cup g)} \subseteq C \cup D\).\\
Let \(y \in C \cup D\), then \(y \in C\) or \(y \in D\).\\
If \(y \in C\), then \({(x,y)} \in h\) for some \(x \in A\), since \(h:A \xrightarrow{onto} 
C\), \(Rng{(h)}=C\).\\
If \(y \in D\), then \({(x,y)} \in g\) for some \(x \in B\), since \(g:B \xrightarrow{onto} D\), \(Rng{(g)}=D\) 
.\\
Thus \({(x,y)} \in h \cup g\), \({y \in Rng{(h \cup g)}}\) and therefore \(C \cup D \subseteq Rng{(h \cup g)}\).\\
Then \(Rng{(h \cup g)} = C \cup D\), since \(y \in C \cup D\) and we have shown there is a \(x \in A\) or \(x \in B\) with \(A \cap B = \varnothing\) such that \({(x,y)} \in h \cup g\).\\
Because we supposed \(h: A \xrightarrow{onto} C\), \(g: B \xrightarrow{onto} D\), \(A \cap B = 
\varnothing\) and let \(y \in C \cup D\), we showed that there is an \(x \in A\) 
or \(x \in B\) such that \({(x,y)} \in h \cup g\) and thus \(Rng{(h \cup g)} = C \cup 
D\).\\
Therefore if \(h:A \xrightarrow{onto} C, g:B\xrightarrow{onto}D, \text{ and } A \cap B = \varnothing\), 
  then \(h \cup g: A \cup B \xrightarrow{onto} C \cup D\).
  \end{proof}
\end{theorem1}
\newpage
\addtocounter{ProblemCounter}{1}
\section*{\S 4.3 Exercise \arabic{ProblemCounter}: Find sets \(A,B,C\) and functions \(f:A \rightarrow B\) and \(g:B \rightarrow C\) such that:}
\setcounter{SubsectionCounter}{1}
\textbf{\arabic{ProblemCounter}.\alph{SubsectionCounter}} \(f\) is onto 
\(B\),but \(g \circ f\) is not onto \(C\).\\
Let \(A = \{1,2,3\}, B=\{a,b,c\}, C=\{x,y,z\}\).\\
\(f:\{{(1,a)},{(2,b)},{(3,c)}\}\), so \(f\) is onto \(B\).\\
\(g:\{{(a,x)},{(b,y)},{(c,y)}\}\), so \(g\) is not onto \(C\).\\
\(g \circ f = \{{(1,x)},{(2,y)},{(3,y)}\}\) so \(g \circ f\) is not onto 
\(C\).\\
\\\addtocounter{SubsectionCounter}{1}
\noindent\textbf{\arabic{ProblemCounter}.\alph{SubsectionCounter}} \(g\) is onto 
\(C\) but \(g \circ f\) is not onto \(C\)\\
Let \(A = \{1,2,3\}, B=\{a,b,c\}, C=\{x,y,z\}\).\\
\(f:\{{(1,a)},{(2,b)},{(3,b)}\}\), so \(f\) is not onto \(B\).\\
\(g:\{{(a,x)},{(b,y)},{(c,z)}\}\), so \(g\) is onto \(C\).\\
\(g \circ f = \{{(1,x)},{(2,y)},{(3,y)}\}\) so \(g \circ f\) is not onto 
\(C\).\\
\\\addtocounter{SubsectionCounter}{1}
\noindent\textbf{\arabic{ProblemCounter}.\alph{SubsectionCounter}} \(g \circ f\) 
is onto \(C\), but \(f\) is not onto \(B\).\\
Let \(A = \{1,2,3\}, B=\{a,b,c,d\}, C=\{x,y,z\}\).\\
\(f:\{{(1,a)},{(2,b)},{(3,c)}\}\), so \(f\) is onto \(B\).\\
\(g:\{{(a,x)},{(b,y)},{(c,z)},{(d,z)}\}\), so \(g\) is onto \(C\).\\
\(g \circ f = \{{(1,x)},{(2,y)},{(3,z)}\}\) so \(g \circ f\) is not onto 
\(C\).\\
\\\addtocounter{SubsectionCounter}{1}
\noindent\textbf{\arabic{ProblemCounter}.\alph{SubsectionCounter}} \(f\) is 
one-to-one, but \(g \circ f\) is not one-to-one.\\
Let \(A = \{1,2,3\}, B=\{a,b,c\}, C=\{x,y,z\}\).\\
\(f:\{{(1,a)},{(2,b)},{(3,c)}\}\), so \(f\) is one-to-one.\\
\(g:\{{(a,x)},{(b,y)},{(c,y)}\}\), so \(g\) is not one-to-one.\\
\(g \circ f = \{{(1,x)},{(2,y)},{(3,y)}\}\) so \(g \circ f\) is not one-to-one.
\newpage
\addtocounter{SubsectionCounter}{1}
\noindent\textbf{\arabic{ProblemCounter}.\alph{SubsectionCounter}} \(g\) is 
one-to-one, but \(g \circ f\) is not one-to-one.\\
Let \(A = \{1,2,3\}, B=\{a,b,c\}, C=\{x,y,z\}\).\\
\(f:\{{(1,a)},{(2,b)},{(3,b)}\}\), so \(f\) is not one-to-one.\\
\(g:\{{(a,x)},{(b,y)},{(c,z)}\}\), so \(g\) is one-to-one.\\
\(g \circ f = \{{(1,x)},{(2,y)},{(3,y)}\}\) so \(g \circ f\) is not 
one-to-one.\\
\\\addtocounter{SubsectionCounter}{1}
\noindent\textbf{\arabic{ProblemCounter}.\alph{SubsectionCounter}} \(g \circ f\) is 
one-to-one, but \(g\) is not one-to-one.\\
Let \(A = \{1,2,3\}, B=\{a,b,c,d\}, C=\{x,y,z\}\).\\
\(f:\{{(1,a)},{(2,b)},{(3,d)}\}\), so \(f\) is not one-to-one.\\
\(g:\{{(a,x)},{(b,y)},{(c,z)},{(d,z)}\}\), so \(g\) is not one-to-one.\\
\(g \circ f = \{{(1,x)},{(2,y)},{(3,z)}\}\) so \(g \circ f\) is one-to-one.
\newpage
\setcounter{ProblemCounter}{4}
\section*{\S 4.4 Exercise \arabic{ProblemCounter}: Prove parts {(b)} of Theorem 4.4.2: If \(F:A \rightarrow B\) and \(F^{-1}\) is a function, then \(F\) is one-to-one}
\setcounter{SubsectionCounter}{1}
\textbf{\arabic{ProblemCounter}.\alph{SubsectionCounter}}
\begin{theorem1}
  Let \(F\) be a function from set \(A\) to set \(B\). If \(F^{-1}\) is a 
  function, then \(F^{-1}\) is one-to-one.
  \begin{proof}
Suppose that \(F: A \rightarrow B\). Let arbitrary \(x,y \in B\).\\
Assume that \(F^{-1}\) is a function and assume that \(F^{-1}{(x)} = F^{-1}{(y)} = z\) for some \(z \in 
A\).\\
Then \({(x,z)} \in F^{-1}\) and \((y,z) \in F^{-1}\).\\
As \(F\) is a function, \(F{(z)}=x\) and \(F{(z)}=y\). Then \(x = y\).\\
Therefore \(F^{-1}\) is one-to-one.\\
Since we supposed \(F:A \rightarrow B\), \(F^{-1}\) is a function, and let an arbitrary \(x,y \in B\) such that \(F^{-1}{(x)} = F^{-1}{(y)} = z\) for some \(z \in 
A\), we showed that if \(F^{-1}\) is a function, then \(F^{-1}\) is one-to-one.
  \end{proof}
\end{theorem1}
\end{document}